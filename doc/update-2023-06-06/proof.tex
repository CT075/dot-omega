\documentclass[a4paper, 10pt]{article}
\usepackage{amssymb, amsmath, amsthm}
\usepackage[backend=bibtex,natbib=true]{biblatex}
\usepackage{enumitem}
\usepackage{mathrsfs}
\usepackage{parskip}
\usepackage{mathpartir}
\usepackage{bussproofs}
\usepackage{stmaryrd}
\usepackage{appendix}
\usepackage{datetime}
\usepackage{tabularx}
\usepackage{xcolor}
\usepackage{pifont}
\usepackage[iso, english]{isodate}
\usepackage[section]{placeins}

\newcommand{\xmark}{\ding{55}}

\newcommand{\DOTw}{\ensuremath{DOT^\omega}}
\newcommand{\Fwint}{\ensuremath{F^\omega_{..}}}
\newcommand{\Dsub}{\ensuremath{D_{<:}}}
\newcommand{\interval}[3][]{#2 .._{#1} #3}
\newcommand{\isctx}[1]{#1\ \texttt{ctx}}
\newcommand{\istctx}[1]{#1\ \texttt{ctx}_\#}
\newcommand{\iskd}[1]{#1\ \texttt{kd}}
\newcommand{\TyKd}{*}
\newcommand{\KDepArr}[3]{\Pi(#1:#2).#3}
\newcommand{\TDepArr}[3]{(#1:#2) \rightarrow #3}
\newcommand{\subst}[3]{#1[#2/#3]}
\newcommand{\objtyp}[3]{\{ \textbf{#1}\ #2 : #3 \}}
\newcommand{\objval}[3]{\{ \textbf{#1}\ #2 = #3 \}}
\newcommand{\termlet}[3]{\text{let }#1 = #2\text{ in }#3}
\newcommand{\envlet}[2]{\text{let }#1 = #2}
\newcommand{\dom}[1]{\textrm{domain}(#1)}
\newcommand{\KDenot}[1]{\llbracket #1 \rrbracket}
\newcommand{\KDenotE}[1]{\mathcal{E} \llbracket #1 \rrbracket}
\newcommand{\invdash}{\vdash_{\#\#}}

\newtheorem{theorem}{Theorem}
\newtheorem{lemma}{Lemma}

\theoremstyle{definition}
\newtheorem{defn}{Definition}
\newtheorem{property}{Property}

\bibliography{paper}

\setlength{\parindent}{0cm}

\title{Soundness for \DOTw{} (update \today)}
\author{Cameron Wong}

\begin{document}
\maketitle

\setlength{\parskip}{\baselineskip}

\section{Last time}

The key to this proof of soundness is in the equivalence of DOT's regular
typing rules to a restricted set of \emph{tight typing} rules which are far
better behaved. Previously, we demonstrated how tight typing can be generalized
to a higher-kinded setting by adjusting the beta-reduction rules to enforce
that the lambda body is still well-kinded with the concrete argument, avoiding
the impossible reductions that are usually permitted by bad bounds.

\section{Normalization of types}

The interesting change in showing type soundness in a higher-kinded
system is in the proof of progress, which requires deducing the form of a value
from its type. Specifically, the type of a term may be itself a complex beta
redex that is equivalent to one with a canonical form. To resolve this, we must
first show that every well-kinded type can indeed be reduced to such a form.

An important observation is that, in inert contexts, all abstract types (type
members and type variables) must be singleton kinds. This means that an inert
context can also serve as a variable store for type-level evaluation (via Lemma
\ref{lemma:typesubst}), meaning we can use a logical-relation based approach
using inert contexts $\Gamma$ as evaluation heaps.

A straightforward attempt may look like so:
\begin{align*}
  \KDenot{\interval{S}{U}} &=& \{ \langle \Gamma, \tau \rangle :
    \Gamma\ \text{inert} \land \Gamma \vdash S \le \tau \le U : * \} \\
  \KDenot{\KDepArr{X}{J}{K}} &=& \{ \langle \Gamma, \lambda(X:J).A \rangle :
    \forall B \in \KDenot{J} .
    \Gamma\ \text{inert} \land
    \langle (\Gamma, X:S(B:K)), A \rangle \in \KDenotE{K} \} \\
  \KDenotE{K} &=& \{ \langle \Gamma, A \rangle :
    \exists V .
    V\ \text{normal} \land
    \Gamma \vdash A == V : K \}
\end{align*}

However, with this definition, it is not true that $\Gamma \vdash \lambda
(X:J).A : \KDepArr{X}{J}{K}$ implies that $\langle \Gamma, \lambda(X:J).A
\rangle \in \KDenotE{\KDepArr{X}{J}{K}}$ when $\Gamma$ is inert. The culprit
is the possibility of bad bounds in the parameter kind $J$, leading to the
evaluation of the function body getting stuck with a type operator of the wrong
form.

\textcolor{red}{I'm 95\% sure that this doesn't actually break the statement --
in order to reach kind $*$, we \emph{must} have some witness that the kind $J$
is good, so it's just a matter of designing the relation to pass this evidence
through properly, but I wasn't able to come up with a good one.}

\textcolor{red}{%
We could attempt to use the same method as Tiark (indexing the relation with
upper and lower bounds), but it would likely require lots of reworking -- in
System D-sub, the bounds are a different syntactic sort (types) as the thing
being reduced (terms), but here the bounds are actually the same sort (types
vs kinds). It also is not obvious to me how to generalize it to function
kinds.}

\section{The Rest}

If types can be strongly normalized, then we can recover \citet{rapoport2017}'s
proofs of progress and preservation by simply normalizing types before
proceeding with the same induction. Importantly, as progress and preservation
are expressed from an empty context, this can always be done by using the
concrete good-bounds witnesses to construct the necessary inert contexts.

If it can't be shown that types are strongly normalizing, we may be able to
get away with weak-head normalization, as that's all that's really necessary
for canonical forms. This will likely require a more involved progress and
preservation proof that interleaves weak-head steps with progress.

\newpage
\appendix

\section{$\DOTw$ Full rules}\label{appendix:dotw-rules}

\begin{figure}[ht]
  \begin{mathpar}
    \inferrule*{\phantom{}}{\isctx{\varnothing}} \and
    \inferrule*{\isctx{\Gamma}\and\Gamma \vdash \iskd{K}}
      {\isctx{\Gamma, X:K}} \and
    \inferrule*{\isctx{\Gamma}\and\Gamma \vdash A : \TyKd}
      {\isctx{\Gamma, x:A}} \and
  \end{mathpar}
  \caption{Context formation}
\end{figure}

\begin{figure}[ht]
  \begin{mathpar}
    \inferrule*[right=Wf-Intv]
      {\Gamma \vdash S : \TyKd \and \Gamma \vdash U : \TyKd}
      {\Gamma \vdash \iskd{\interval{S}{U}}} \and
    \inferrule*[right=Wf-DArr]
      {\Gamma \vdash \iskd{J} \and \Gamma, X:J \vdash \iskd{K}}
      {\Gamma \vdash \iskd{\KDepArr{X}{J}{K}}}
  \end{mathpar}
  \caption{Kind formation}
\end{figure}

\begin{figure}[ht]
  \begin{mathpar}
    \inferrule*[right=SK-Intv]
      {\Gamma \vdash S_2 \le S_1:\TyKd \and \Gamma \vdash U_1 \le U_2:\TyKd}
      {\Gamma \vdash \interval{S_1}{U_1} \le \interval{S_2}{U_2}} \and
    \inferrule*[right=SK-DArr]
      {\Gamma \vdash \iskd{\KDepArr{X}{J_1}{K_1}} \and
       \Gamma \vdash J_2 \le J_1 \and
       \Gamma, X:J_2 \vdash K_1 \le K_2}
      {\Gamma \vdash \KDepArr{X}{J_1}{K_1} \le \KDepArr{X}{J_2}{K_2}}
  \end{mathpar}
  \caption{Subkinding}
\end{figure}

\begin{figure}[ht]
  \begin{mathpar}
    \inferrule*[right=K-Var]{\isctx{\Gamma, X:K}}{\Gamma, X:K \vdash X:K} \and
    \inferrule*[right=K-Top]{\phantom{}}{\Gamma \vdash \top : \TyKd} \and
    \inferrule*[right=K-Bot]{\phantom{}}{\Gamma \vdash \bot : \TyKd} \and
    \inferrule*[right=K-Sing]{\Gamma \vdash A : \interval{S}{U}}
      {\Gamma \vdash A : \interval{A}{A}} \and
    \inferrule*[right=K-Arr]
      {\Gamma \vdash A : \TyKd \and \Gamma, x : A \vdash B : \TyKd}
      {\Gamma \vdash \TDepArr{x}{A}{B} : \TyKd} \and
    \inferrule*[right=K-Abs]
      {\Gamma \vdash \iskd{J} \and \Gamma, X:J \vdash A : K \and
       \Gamma, X:J \vdash \iskd{K}}
      {\Gamma \vdash \lambda(X:J).A : \KDepArr{X}{J}{K}} \and
    \inferrule*[right=K-App]
      {\Gamma \vdash A : \KDepArr{X}{J}{K} \and \Gamma \vdash B : J \and
       \Gamma, X:J \vdash \iskd{K} \and \Gamma \vdash \iskd{\subst{K}{B}{X}}}
      {\Gamma \vdash A\ B : \subst{K}{B}{X}} \and
    \inferrule*[right=K-Intersect]
      {\Gamma \vdash A : \interval{S_1}{U_1} \and
       \Gamma \vdash B : \interval{S_2}{U_2}}
      {\Gamma \vdash A \land B : \interval{S_1 \lor S_2}{U_1 \land U_2}} \and
    \inferrule*[right=K-Field]
      {\Gamma \vdash A : \interval{S}{U}}
      {\Gamma \vdash \objtyp{val}{\ell}{A} : *} \and
    \inferrule*[right=K-Typ]
      {\Gamma \vdash \iskd{K}}
      {\Gamma \vdash \objtyp{type}{M}{K} : *} \and
    \inferrule*[right=K-Typ-Mem]
      {\Gamma \vdash x : \objtyp{type}{M}{K}}
      {\Gamma \vdash x.M : K} \and
    \inferrule*[right=K-Rec]
      {\Gamma, x : \tau \vdash \tau : K}
      {\Gamma \vdash \mu(x . \tau) : K} \and
    \inferrule*[right=K-Sub]
      {\Gamma \vdash A:J \and \Gamma \vdash J \le K}{\Gamma \vdash A:K}
  \end{mathpar}
  \caption{Kind assignment}
\end{figure}
\clearpage

Note that \textsc{K-Intersect} rules refers to the union type $S_1 \lor S_2$,
despite no such construct being present in the language as a whole. I am
currently investigating whether the explicit addition of this construct is
necessary.

\begin{figure}[ht]
  \begin{mathpar}
    \inferrule*[right=ST-refl]{\Gamma \vdash A : K}
      {\Gamma \vdash A \le A : K} \and
    \inferrule*[right=ST-trans]
      {\Gamma \vdash A \le B : K \and
       \Gamma \vdash B \le C : K}
      {\Gamma \vdash A \le C : K} \and
    \inferrule*[right=ST-top]{\Gamma \vdash A : \interval{S}{U}}
      {\Gamma \vdash A \le \top : \TyKd} \and
    \inferrule*[right=ST-bot]{\Gamma \vdash A : \interval{S}{U}}
      {\Gamma \vdash \bot \le A : \TyKd} \and
    \inferrule*[right=ST-and-$\ell_1$]
      {\Gamma \vdash A \land B : K}{\Gamma \vdash A \land B \le A : K} \and
    \inferrule*[right=ST-and-$\ell_2$]
      {\Gamma \vdash A \land B : K}{\Gamma \vdash A \land B \le B : K} \and
    \inferrule*[right=ST-and-r]
      {\Gamma \vdash S \le A : K \and \Gamma \vdash S \le B : K}
      {\Gamma \vdash S \le A \land B : K} \and
    \inferrule*[right=ST-field]{\Gamma \vdash A \le B : \TyKd}
      {\Gamma \vdash \objtyp{val}{\ell}{A} \le \objtyp{val}{\ell}{B} : \TyKd}
      \and
    \inferrule*[right=ST-typ]
      {\Gamma \vdash J \le K}
      {\Gamma \vdash \objtyp{type}{M}{J} \le \objtyp{type}{M}{K} : \TyKd}\and
    \inferrule*[right=ST-$\beta_1$]
      {\Gamma Z : J \vdash A : K \and
       \Gamma \vdash B : J}
      {\Gamma \vdash (\lambda (Z: J).A)\ B \le \subst{A}{Z}{B} : \subst{K}{Z}{B}}
      \and
    \inferrule*[right=ST-$\beta_2$]
      {\Gamma Z : J \vdash A : K \and
       \Gamma \vdash B : J}
      {\Gamma \vdash \subst{A}{Z}{B} \le (\lambda (Z: J).A)\ B : \subst{K}{Z}{B}}
  \end{mathpar}
  \caption{Subtyping}
\end{figure}

\begin{figure}[ht]
  \begin{mathpar}
    \inferrule*[right=Eq]
      {\Gamma \vdash A \le B : K \and \Gamma \vdash B \le A : K}
      {\Gamma \vdash A = B : K}
  \end{mathpar}
  \caption{Type equality}
\end{figure}

\begin{figure}[ht]
  \begin{mathpar}
    \inferrule*[right=Ty-Var]
      {\isctx{\Gamma, x: \tau}}
      {\Gamma, x:\tau \vdash x: \tau} \and
    \inferrule*[right=Ty-Let]
      {\Gamma \vdash e_1: \tau \and \Gamma, x:\tau \vdash e_2 : \rho \and x \not\in fv(\rho)}
      {\Gamma \vdash \termlet{x}{e_1}{e_2} : \rho} \and
    \inferrule*[right=Ty-Fun-I]
      {\Gamma, x: \tau \vdash e: \rho}
      {\Gamma \vdash \lambda(x:\tau).e : \TDepArr{x}{\tau}{\rho}} \and
    \inferrule*[right=Ty-Fun-E]
      {\Gamma \vdash x : \TDepArr{z}{\tau}{\rho} \and \Gamma \vdash y : \tau}
      {\Gamma \vdash x\ y : \subst{\rho}{z}{y}} \and
    \inferrule*[right=Ty-$\mu$-I]
      {\Gamma \vdash x: \tau}
      {\Gamma \vdash x: \mu(x: \tau)} \and
    \inferrule*[right=Ty-$\mu$-E]
      {\Gamma \vdash x: \mu(z: \tau)}
      {\Gamma \vdash x: \subst{\tau}{z}{x}} \and
    \inferrule*[right=Ty-Rec-I]
      {\Gamma, x: \tau \vdash d: \tau}
      {\Gamma \vdash \nu(x: \tau)d : \mu(x:\tau)} \and
    \inferrule*[right=Ty-Rec-E]
      {\Gamma, x: \objtyp{val}{\ell}{\tau}}
      {\Gamma \vdash x.\ell : \tau} \and
    \inferrule*[right=Ty-And-I]
      {\Gamma \vdash x: \tau_1 \and \Gamma \vdash x: \tau_2}
      {\Gamma \vdash x: \tau_1 \land \tau_2} \and
    \inferrule*[right=Ty-Sub]
      {\Gamma \vdash e: \tau_1 \and \Gamma \vdash \tau_1 \le \tau_2 : *}
      {\Gamma \vdash e: \tau_2} \and
    \inferrule*[right=Ty-Def-Trm]
      {\Gamma \vdash e: \rho}
      {\Gamma \vdash \objval{val}{\ell}{e} : \objtyp{val}{\ell}{\rho}} \and
    \inferrule*[right=Ty-Def-Typ]
      {\Gamma \vdash \tau : K}
      {\Gamma \vdash \objval{type}{M}{A} : \objtyp{type}{M}{S(A:K)}}
  \end{mathpar}
  \caption{Type assignment}
\end{figure}

\section{Tight typing rules}\label{appendix:tight-rules}

In most cases, tight typing is merely forwarded to the premises. In any rule
that extends the context with possibly-untrusted bounds, tight typing reverts
to general typing.

\begin{figure}[ht]
  \begin{mathpar}
    \inferrule*{\phantom{}}{\istctx{\varnothing}} \and
    \inferrule*{\istctx{\Gamma}\and\Gamma \vdash_\# \iskd{K}}
      {\isctx{\Gamma, X:K}} \and
    \inferrule*{\istctx{\Gamma}\and\Gamma \vdash_\# A : \TyKd}
      {\isctx{\Gamma, x:A}} \and
  \end{mathpar}
  \caption{Context formation}
\end{figure}

\begin{figure}[ht]
  \begin{mathpar}
    \inferrule*[right=Wf-Int-\#]
      {\Gamma \vdash_\# S : \TyKd \and \Gamma \vdash_\# U : \TyKd}
      {\Gamma \vdash_\# \iskd{\interval{S}{U}}} \and
    \inferrule*[right=Wf-DArr-\#]
      {\Gamma \vdash_\# \iskd{J} \and \Gamma, X:J \vdash_\# \iskd{K}}
      {\Gamma \vdash_\# \iskd{\KDepArr{X}{J}{K}}}
  \end{mathpar}
  \caption{Kind formation}
\end{figure}

\begin{figure}[ht]
  \begin{mathpar}
    \inferrule*[right=SK-Intv-\#]
      {\Gamma \vdash_\# S_2 \le S_1:\TyKd \and \Gamma \vdash_\# U_1 \le U_2:\TyKd}
      {\Gamma \vdash_\# \interval{S_1}{U_1} \le \interval{S_2}{U_2}} \and
    \inferrule*[right=SK-DArr-\#]
      {\Gamma \vdash_\# \iskd{\KDepArr{X}{J_1}{K_1}} \and
       \Gamma \vdash_\# J_2 \le J_1 \and
       \Gamma, X:J_2 \vdash K_1 \le K_2}
      {\Gamma \vdash_\# \KDepArr{X}{J_1}{K_1} \le \KDepArr{X}{J_2}{K_2}}
  \end{mathpar}
  \caption{Subkinding}
\end{figure}

\begin{figure}[ht]
  \begin{mathpar}
    \inferrule*[right=K-Var-\#]{\istctx{\Gamma, X:K}}{\Gamma, X:K \vdash_\# X:K} \and
    \inferrule*[right=K-Top-\#]{\phantom{}}{\Gamma \vdash_\# \top : \TyKd} \and
    \inferrule*[right=K-Bot-\#]{\phantom{}}{\Gamma \vdash_\# \bot : \TyKd} \and
    \inferrule*[right=K-Sing-\#]{\Gamma \vdash_\# A : \interval{S}{U}}
      {\Gamma \vdash_\# A : \interval{A}{A}} \and
    \inferrule*[right=K-Arr-\#]
      {\Gamma \vdash_\# A : \TyKd \and \Gamma, x : A \vdash B : \TyKd}
      {\Gamma \vdash_\# \TDepArr{x}{A}{B} : \TyKd} \and
    \inferrule*[right=K-Abs-\#]
      {\Gamma \vdash_\# \iskd{J} \and \Gamma, X:J \vdash A : K \and
       \Gamma, X:J \vdash_\# \iskd{K}}
      {\Gamma \vdash_\# \lambda(X:J).A : \KDepArr{X}{J}{K}} \and
    \inferrule*[right=K-App-\#]
      {\Gamma \vdash_\# A : \KDepArr{X}{J}{K} \and \Gamma \vdash_\# B : J \and
       \Gamma, X:J \vdash \iskd{K} \and \Gamma \vdash_\# \iskd{\subst{K}{B}{X}}}
      {\Gamma \vdash_\# A\ B : \subst{K}{B}{X}} \and
    \inferrule*[right=K-Intersect-\#]
      {\Gamma \vdash_\# A : \interval{S_1}{U_1} \and
       \Gamma \vdash_\# B : \interval{S_2}{U_2}}
      {\Gamma \vdash_\# A \land B : \interval{S_1 \lor S_2}{U_1 \land U_2}} \and
    \inferrule*[right=K-Field-\#]
      {\Gamma \vdash_\# A : \interval{S}{U}}
      {\Gamma \vdash_\# \objtyp{val}{\ell}{A} : *} \and
    \inferrule*[right=K-Typ-\#]
      {\Gamma \vdash_\# \iskd{K}}
      {\Gamma \vdash_\# \objtyp{type}{M}{K} : *} \and
    \inferrule*[right=K-Typ-Mem-\#]
      {\Gamma \vdash_! x : \objtyp{type}{M}{S(A:K)}}
      {\Gamma \vdash_\# x.M : S(A:K)} \and
    \inferrule*[right=K-Rec-\#]
      {\Gamma, x : \tau \vdash \tau : K}
      {\Gamma \vdash_\# \mu(x . \tau) : K} \and
    \inferrule*[right=K-Sub-\#]
      {\Gamma \vdash_\# A:J \and \Gamma \vdash_\# J \le K}{\Gamma \vdash_\# A:K}
  \end{mathpar}
  \caption{Kind assignment}
\end{figure}

\begin{figure}[ht]
  \begin{mathpar}
    \inferrule*[right=ST-refl-\#]{\Gamma \vdash_\# A : K}
      {\Gamma \vdash_\# A \le A : K} \and
    \inferrule*[right=ST-trans-\#]
      {\Gamma \vdash_\# A \le B : K \and
       \Gamma \vdash_\# B \le C : K}
      {\Gamma \vdash_\# A \le C : K} \and
    \inferrule*[right=ST-top-\#]{\Gamma \vdash_\# A : \interval{S}{U}}
      {\Gamma \vdash_\# A \le \top : \TyKd} \and
    \inferrule*[right=ST-bot-\#]{\Gamma \vdash_\# A : \interval{S}{U}}
      {\Gamma \vdash_\# \bot \le A : \TyKd} \and
    \inferrule*[right=ST-and-$\ell_1$-\#]
      {\Gamma \vdash_\# A \land B : K}{\Gamma \vdash_\# A \land B \le A : K} \and
    \inferrule*[right=ST-and-$\ell_2$-\#]
      {\Gamma \vdash_\# A \land B : K}{\Gamma \vdash_\# A \land B \le B : K} \and
    \inferrule*[right=ST-and-r-\#]
      {\Gamma \vdash_\# S \le A : K \and \Gamma \vdash_\# S \le B : K}
      {\Gamma \vdash_\# S \le A \land B : K} \and
    \inferrule*[right=ST-field-\#]{\Gamma \vdash_\# A \le B : \TyKd}
      {\Gamma \vdash_\# \objtyp{val}{\ell}{A} \le \objtyp{val}{\ell}{B} : \TyKd}
      \and
    \inferrule*[right=ST-typ-\#]
      {\Gamma \vdash_\# J \le K}
      {\Gamma \vdash_\# \objtyp{type}{M}{J} \le \objtyp{type}{M}{K} : \TyKd}\and
    \inferrule*[right=ST-$\beta_1$-\#]
      {\Gamma \vdash_\# B : J \and
       \Gamma, Z: S(B:J) \vdash_\# A : K}
      {\Gamma \vdash_\# (\lambda (Z: J).A)\ B \le \subst{A}{Z}{B} : \subst{K}{Z}{B}}
      \and
    \inferrule*[right=ST-$\beta_2$-\#]
      {\Gamma \vdash_\# B : J \and
       \Gamma, Z: S(B:J) \vdash_\# A : K}
      {\Gamma \vdash_\# \subst{A}{Z}{B} \le (\lambda (Z: J).A)\ B : \subst{K}{Z}{B}}
  \end{mathpar}
  \caption{Subtyping}
\end{figure}

\begin{figure}[ht]
  \begin{mathpar}
    \inferrule*[right=Eq-\#]
      {\Gamma \vdash_\# A \le B : K \and \Gamma \vdash_\# B \le A : K}
      {\Gamma \vdash_\# A = B : K}
  \end{mathpar}
  \caption{Type equality}
\end{figure}

\begin{figure}[ht]
  \begin{mathpar}
    \inferrule*[right=Ty-Var-\#]
      {\istctx{\Gamma, x: \tau}}
      {\Gamma, x:\tau \vdash_\# x: \tau} \and
    \inferrule*[right=Ty-Let-\#]
      {\Gamma \vdash_\# e_1: \tau \and \Gamma, x:\tau \vdash e_2: \rho \and x \not\in fv(\rho)}
      {\Gamma \vdash_\# \termlet{x}{e_1}{e_2} : \rho} \and
    \inferrule*[right=Ty-Fun-I-\#]
      {\Gamma, x: \tau \vdash e: \rho}
      {\Gamma \vdash_\# \lambda(x:\tau).e : \TDepArr{x}{\tau}{\rho}} \and
    \inferrule*[right=Ty-Fun-E-\#]
      {\Gamma \vdash_\# x : \TDepArr{z}{\tau}{\rho} \and \Gamma \vdash_\# y : \tau}
      {\Gamma \vdash_\# x\ y : \subst{\rho}{z}{y}} \and
    \inferrule*[right=Ty-$\mu$-I-\#]
      {\Gamma \vdash_\# x: \tau}
      {\Gamma \vdash_\# x: \mu(x: \tau)} \and
    \inferrule*[right=Ty-$\mu$-E-\#]
      {\Gamma \vdash_\# x: \mu(z: \tau)}
      {\Gamma \vdash_\# x: \subst{\tau}{z}{x}} \and
    \inferrule*[right=Ty-Rec-I-\#]
      {\Gamma, x: \tau \vdash d: \tau}
      {\Gamma \vdash_\# \nu(x: \tau)d : \mu(x:\tau)} \and
    \inferrule*[right=Ty-Rec-E-\#]
      {\Gamma, x: \objtyp{val}{\ell}{\tau}}
      {\Gamma \vdash_\# x.\ell : \tau} \and
    \inferrule*[right=Ty-And-I-\#]
      {\Gamma \vdash_\# x: \tau_1 \and \Gamma \vdash_\# x: \tau_2}
      {\Gamma \vdash_\# x: \tau_1 \land \tau_2} \and
    \inferrule*[right=Ty-Sub-\#]
      {\Gamma \vdash_\# e: \tau_1 \and \Gamma \vdash_\# \tau_1 \le \tau_2 : *}
      {\Gamma \vdash_\# e: \tau_2} \and
    \inferrule*[right=Ty-Def-Trm-\#]
      {\Gamma \vdash_\# e: \rho}
      {\Gamma \vdash_\# \objval{val}{\ell}{e} : \objtyp{val}{\ell}{\rho}} \and
    \inferrule*[right=Ty-Def-Typ-\#]
      {\Gamma \vdash_\# \tau : K}
      {\Gamma \vdash_\# \objval{type}{M}{A} : \objtyp{type}{M}{S(A:K)}}
  \end{mathpar}
  \caption{Type assignment}
\end{figure}

\begin{figure}[ht]
  \begin{mathpar}
    \inferrule*[right=Var-!]{\phantom{}}{\Gamma, x:\tau \vdash_! x:\tau}\and
    \inferrule*[right=Rec-E-!]{\Gamma \vdash_! x: \mu(z:\tau)}
      {\Gamma \vdash_! x: \subst{\tau}{z}{x}} \and
    \inferrule*[right=And$_1$-E-!]
      {\Gamma \vdash_! x: \tau_1 \land \tau_2}
      {\Gamma \vdash_! x: \tau_1}\and
    \inferrule*[right=And$_2$-E-!]
      {\Gamma \vdash_! x: \tau_1 \land \tau_2}
      {\Gamma \vdash_! x: \tau_2}\and
    \inferrule*[right=Fun-I-!]
      {\Gamma, x: \tau \vdash e: \rho \and x \not\in fv(\tau)}
      {\Gamma \vdash_! \lambda(x:\tau).e : \TDepArr{x}{\tau}{\rho}} \and
    \inferrule*[right=Record-I-!]
      {\Gamma, x: \tau \vdash d: \tau}
      {\Gamma \vdash_! \nu(x:\tau)d : \mu(x:\tau)}
  \end{mathpar}
  \caption{Precise value and variable typing}
\end{figure}

\section{Invertible Typing}\label{appendix:invertible}

\begin{figure}[ht]
  \begin{mathpar}
    \inferrule*[right=Var-\#\#]{\Gamma \vdash_! x: \tau}{\Gamma \vdash_{\#\#} x: \tau}\and
    \inferrule*[right=Val-\#\#]
      { \Gamma \invdash x: \objtyp{val}{\ell}{\tau} \and
        \Gamma \vdash_\# \tau \le \rho : *}
      { \Gamma \invdash x: \objtyp{val}{\ell}{\rho} } \and
    \inferrule*[right=Type-\#\#]
      { \Gamma \invdash x: \objtyp{type}{M}{J} \and
        \Gamma \vdash_\# J \le K}
      { \Gamma \invdash x: \objtyp{type}{M}{K} } \and
    \inferrule*[right=Fun-\#\#]
      { \Gamma \invdash x: \TDepArr{z}{S}{T} \and
        \Gamma \vdash_\# S' \le S : J \and
        \Gamma, z: S' \vdash_\# T \le T' : K }
      { \Gamma \invdash x : \TDepArr{z}{S'}{T'} } \and
    \inferrule*[right=Intersect-\#\#]
      { \Gamma \invdash x: A \and \Gamma \invdash x: B }
      { \Gamma \invdash x: A \land B } \and
    \inferrule*[right=Sel-\#\#]
      { \Gamma \invdash x: A \and
        \Gamma \vdash_! z : \objtyp{type}{M}{\interval{A}{A}} }
      { \Gamma \invdash x : z.M } \and
    \inferrule*[right=Rec-I-\#\#]
      { \Gamma \invdash x: \tau }
      { \Gamma \invdash x: \mu(x: \tau) } \and
    \inferrule*[right=Top-\#\#]
      { \Gamma \invdash x: \tau }
      { \Gamma \invdash x: \top }
  \end{mathpar}
  \caption{Invertible value and variable typing}
\end{figure}

\section{Auxiliary Lemmas}

\begin{lemma}[Type Substitution]\label{lemma:typesubst}
  For inert contexts $\Gamma$,
  \begin{itemize}
    \item $\Gamma, X: S(A:J) \vdash T : K$ implies
      $\Gamma \vdash \subst{T}{X}{A} : \subst{K}{X}{A}$
    \item $\Gamma, X: S(A:J) \vdash T_1 \le T_2 : K$ implies
      $\Gamma \vdash \subst{T_1}{X}{A} \le \subst{T_2}{X}{A} : \subst{K}{X}{A}$
  \end{itemize}
\end{lemma}
\begin{proof}
  Convert to tight typing, then induct on the tight typing and tight subtyping
  judgments.
\end{proof}

\begin{lemma}[Tight to invertible typing]
  For inert contexts $\Gamma$, $\Gamma \vdash_\# x: \tau$ implies
  $\Gamma \invdash x: \tau$, and for all values $v$, $\Gamma \vdash_\# v: \tau$
  implies $\Gamma \invdash v: \tau$.
\end{lemma}
\begin{proof}
  By straightforward induction on $\Gamma \vdash_\# x: \tau$ and $\Gamma
  \vdash_\# v: \tau$. \textcolor{red}{This is formalized in Agda.}
\end{proof}

\section{$\DOTw{}$ Operational Semantics}

\begin{figure}[ht]
  \begin{mathpar}
    \inferrule*[right=Term]
      {t \mapsto t'}
      {E[t] \mapsto E[t']} \\
    \inferrule*[right=Apply]
      {v = \lambda(z: \tau).t}
      {\termlet{x}{v}{E[x\ y]} \mapsto \termlet{x}{v}{E[\subst{t}{y}{z}]}} \and
    \inferrule*[right=Project]
      {v = \nu(x:\tau)...\objval{val}{\ell}{t}}
      {\termlet{x}{v}{E[x.\ell]} \mapsto \termlet{x}{v}{E[t]}} \\
    \inferrule*[right=Let-Var]
      {\strut}
      {\termlet{x}{y}{t} \mapsto \subst{t}{y}{x}} \and
    \inferrule*[right=Let-Let]
      {\strut}
      {\termlet{x}{(\termlet{y}{e}{t'})}{t} \mapsto \termlet{y}{e}{\termlet{x}{t'}{t}}}
  \end{mathpar}
  \caption{$\DOTw$ Operational Semantics (\citet{amin2016})}
\end{figure}

\begin{figure}[ht]
  \begin{mathpar}
    \inferrule*[right=Apply]
      {E\text{ contains the binding }\envlet{x}{\lambda(z:\tau).t}}
      {E[x\ y] \mapsto E[\subst{t}{y}{z}]} \and
    \inferrule*[right=Project]
      {E\text{ contains the binding }\envlet{x}{\nu(x: \tau)...\objval{val}{\ell}{t}}}
      {E[x.\ell] \mapsto E[t]} \\
    \inferrule*[right=Let-Var]
      {\strut}
      {E[\termlet{x}{[y]}{t}] \mapsto E[\subst{t}{y}{x}]} \and
    \inferrule*[right=Let-Let]
      {\strut}
      {E[\termlet{x}{[\termlet{y}{e}{t'}]}{t}] \mapsto E[\termlet{y}{e}{\termlet{x}{t'}{t}}]}
  \end{mathpar}
  \caption{$\DOTw$ Operational Semantics with inlined \textsc{Term} (\citet{rapoport2017})}
\end{figure}

\end{document}
