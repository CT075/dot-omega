\documentclass[a4paper, 10pt]{article}
\usepackage{amssymb, amsmath, amsthm}
\usepackage[backend=bibtex,natbib=true]{biblatex}
\usepackage{enumitem}
\usepackage{mathrsfs}
\usepackage{parskip}
\usepackage{mathpartir}
\usepackage{bussproofs}
\usepackage{stmaryrd}
\usepackage{appendix}
\usepackage{datetime}
\usepackage{tabularx}
\usepackage{xcolor}
\usepackage{pifont}
\usepackage[iso, english]{isodate}
\usepackage[section]{placeins}

\newcommand{\xmark}{\ding{55}}

\newcommand{\DOTw}{\ensuremath{DOT^\omega}}
\newcommand{\Fwint}{\ensuremath{F^\omega_{..}}}
\newcommand{\Dsub}{\ensuremath{D_{<:}}}
\newcommand{\interval}[3][]{#2 .._{#1} #3}
\newcommand{\isctx}[1]{#1\ \texttt{ctx}}
\newcommand{\istctx}[1]{#1\ \texttt{ctx}_\#}
\newcommand{\iskd}[1]{#1\ \texttt{kd}}
\newcommand{\TyKd}{*}
\newcommand{\KDepArr}[3]{\Pi(#1:#2).#3}
\newcommand{\TDepArr}[3]{(#1:#2) \rightarrow #3}
\newcommand{\subst}[3]{#1[#2/#3]}
\newcommand{\objtyp}[3]{\{ \textbf{#1}\ #2 : #3 \}}
\newcommand{\objval}[3]{\{ \textbf{#1}\ #2 = #3 \}}
\newcommand{\termlet}[3]{\text{let }#1 = #2\text{ in }#3}
\newcommand{\envlet}[2]{\text{let }#1 = #2}
\newcommand{\dom}[1]{\textrm{domain}(#1)}
\newcommand{\KDenot}[1]{\llbracket #1 \rrbracket}
\newcommand{\KDenotE}[1]{\mathcal{E} \llbracket #1 \rrbracket}
\newcommand{\Sing}[2]{Sing_{#2}(#1)}
\newcommand{\invdash}{\vdash_{\#\#}}

\newtheorem{theorem}{Theorem}
\newtheorem{lemma}{Lemma}

\theoremstyle{definition}
\newtheorem{defn}{Definition}
\newtheorem{property}{Property}

\bibliography{paper}

\setlength{\parindent}{0cm}

\title{%
  Weak-head normalization of $\DOTw{}$ types \\
  \large Soundness for \DOTw{} update \today}
\author{Cameron Wong}

\begin{document}
\maketitle

% TODO: Fix up prose errors
% TODO: Add some intuition on how the relation is constructed

\setlength{\parskip}{\baselineskip}

\section{Last time}

Last time, we discussed the generalization of \emph{tight typing}
\citet{rapoport2017} to a higher-kinded setting. Tight typing is a
re-formulation of the full DOT typing rules to have nicer properties. We also
showed that tight typing is equivalent to regular typing under a class of
well-behaved (\emph{inert}) contexts. Specifically, an inert context is one in
which all abstract types (either introduced indirectly via type members or
directly via type-level functions) have \emph{singleton kinds} (fig.
\ref{fig:singletons}). Specifically, this also necessitated strengthening the
premises of type-level beta-reduction to require that the body of an applied
lambda still typechecks after substitution.

\begin{figure}[ht!]
  \begin{align*}
    \Sing{A}{\interval{S}{U}} &= \interval{A}{A} \\
    \Sing{A}{\KDepArr{X}{J}{K}} &= \KDepArr{X}{J}{\Sing{A\ X}{K}}
  \end{align*}
  \caption{Defining singleton kinds}\label{fig:singletons}
\end{figure}

Moving forwards, generalizing \citet{rapoport2017}'s proof of preservation to
$\DOTw{}$ is straightforward, with no issues specific to the higher-kinded
setting. Progress, however, requires canonical forms, which in turn requires at
least some level of evaluation of type operators. We show only weak-head
normalization, which is enough to give us back our canonical forms lemmas.

\section{Inert contexts as variable stores}

Typically, a logical relations-based proof requires tracking a variable store,
with extra machinery ensuring that the store remains in sync with the typing
context. Attempting to do so for type-level reductions in $\DOTw{}$, however,
is surprisingly fraught, owing to $\DOTw{}$'s type language making use of two
non-interchangeable classes of syntactic variables (namely, type variables and
term variables), only one of which is ever substituted for. Splitting the
context to assign to type and term variables separately is possible
but does not actually solve the problem, as we are still unable to reduce
applications of the form $x.F\ A$ for some termvar $x$.

A second problem is that, just by the nature of the high-level definition of
interval kinds, any denotation $\KDenot{\interval{S}{U}}$ must necessarily
reference the typing context $\Gamma$ to even state that $\Gamma \vdash_\# S
\le \tau \le U$.

Luckily, inert contexts provide a way out. We make use of the following
property, which follows directly from the definition of inertness:
\begin{property}
  For inert contexts $\Gamma$, typevars $X$ and termvars $x$:
  \begin{itemize}
    \item if $\Gamma(X) = K$, or
    \item if $\Gamma(x) : \mu(\dots \land \objval{type}{M}{K} \land \dots)$,
  \end{itemize}
  then $K = \Sing{A}{K'}$ for some $A$ and $K'$.
\end{property}

We also make use of the following lemmas:
\begin{lemma}\label{lemma:sing-defn}
  If $\Gamma \vdash_\# A : \Sing{B}{K}$, then $\Gamma \vdash_\# A == B : K$
\end{lemma}
\begin{proof}
  By induction on the kind $K$. If $K = \interval{S}{U}$, then $\Sing{B}{K}=
  \interval{B}{B}$, so $\Gamma \vdash_\# B \le A \le B : \interval{S}{U}$ by
  \textsc{ST-bnd}$_1$ and \textsc{ST-bnd}$_2$.
\end{proof}
\begin{lemma}[Type Substitution]\label{lemma:typesubst}
  For inert contexts $\Gamma$,
  \begin{itemize}
    \item $\Gamma, X: \Sing{A}{J} \vdash T : K$ implies
      $\Gamma \vdash \subst{T}{X}{A} : \subst{K}{X}{A}$
    \item $\Gamma, X: \Sing{A}{J} \vdash T_1 \le T_2 : K$ implies
      $\Gamma \vdash \subst{T_1}{X}{A} \le \subst{T_2}{X}{A} : \subst{K}{X}{A}$
  \end{itemize}
\end{lemma}
\begin{proof}
  Convert to tight typing, then induct on the tight typing and tight subtyping
  judgments.
\end{proof}

Together, these mean that inert contexts can actually serve as (type) variable
stores themselves. While this does require additional bookkeeping to ensure
that the context remains inert while traversing the (tight) typing derivation,
tight typing is designed precisely to ensure this. This has the additional
side-effect of forcing us to reason entirely in terms of tight typing, but we
wanted to do that anyway\footnote{Arbitrary reduction of open types is already
known to be unsound in unrestricted DOT-style calculi. While we may be able to
recover (weak-head) normalization anyway by starting from an empty context, any
proof based on the unrestricted rules will need some mechanism to address
abstract type members anyway, which was the original point of the tight typing
rules to begin with.}.

\section{The Proof}

We define our logical relation as follows:
\begin{align*}
  \KDenot{\interval{S}{U}} &=& \{ \langle \Gamma, \tau \rangle :
    \tau\ \text{whnf} \land
    \Gamma \vdash_\# S \le \tau \le U : \TyKd \} \\
  \KDenot{\KDepArr{X}{J}{K}} &=& \{ \langle \Gamma, \lambda(X:J).A \rangle :
    % TODO: Tighten up the notation -- it should be <Gamma, B>, not just B
    \forall B \in \KDenotE{J} .
    \langle (\Gamma, X:\Sing{B}{J}), A \rangle \in \KDenotE{K} \} \\
  \KDenotE{K} &=& \{ \langle \Gamma, A \rangle :
    \exists V .
    V\ \text{whnf} \land
    \Gamma \vdash_\# A == V : K \}
\end{align*}

In a standard termination relation featuring a variable heap $H$, there is
typically a secondary consistency premise $\Gamma \vDash H$ stating that
$\Gamma(x) = T$ implies $H(x) \in \KDenot{T}$. Similarly, we need a side
condition that, if $\Gamma(X) = \Sing{A}{K}$, then $\langle \Gamma, A \rangle
\in \KDenotE{\Sing{A}{K}}$, which we will write as $\Gamma\ store$.

We also make use of the following structural lemmas:
\begin{lemma}[Relation Inversion]\label{lemma:inv}
  \begin{mathpar}
    \inferrule*
      {\langle \Gamma, A \rangle \in \KDenotE{K}}
      {\Gamma \vdash_\# A : K}
  \end{mathpar}
\end{lemma}
\begin{proof}
  Falls out of $\Gamma \vdash_\# A == \tau : K$.
\end{proof}

\begin{lemma}[Relation Substitution]\label{lemma:rel-subst}
  If $\langle (\Gamma, X:\Sing{B}{J}), A \rangle \in \KDenotE{K}$, then
  $\langle \Gamma, \subst{A}{X}{B} \in \subst{K}{X}{B} \rangle$.
\end{lemma}
\begin{proof}
  \textcolor{red}{Actually, this feels a little suspect. It seems like it must
  be true, but my attempts to prove it ends up being circular with the main
  theorem. I think there may be some termination measure I can use that is
  preserved by kind substitution.}
\end{proof}

At last, the strong normalization theorem, then, is as follows:

\begin{theorem}[Weak-head normalization of types]
  \begin{mathpar}
    \inferrule*
      {\Gamma \vdash_\# A : K \and K\ store}
      {\langle \Gamma, A \rangle \in \KDenotE{K}}
  \end{mathpar}
\end{theorem}
\begin{proof}
  By induction on the derivation:

  % TODO: Mention that these are only the interesting cases.
  \begin{itemize}
    \item Case \textsc{K-Var-\#}: By the store property.

    \item Case \textsc{K-App-\#}:

      The relevant premises are $\Gamma \vdash_\# A: \KDepArr{X}{J}{K}$ and
      $\Gamma \vdash_\# B: J$. By the inductive hypotheses, we have that
      there exist some $J'$, $A'$ such that $A = \lambda (X:J').A'$ and
      for all $B' \in \KDenotE{J}$, $\langle (\Gamma, X:\Sing{B'}{J}), A'
      \rangle \in \KDenotE{K}$. Instantiating with $B$ gives $\langle (\Gamma,
      X:\Sing{B}{J}), A' \rangle \in \KDenotE{K}$. By Lemma \ref{lemma:inv}, we
      get that $\Gamma, X:\Sing{B}{J} \vdash_\# A' \in K$, providing the
      necessary premise for the \textsc{ST-$\beta$} rules, giving
      $\Gamma \vdash_\# A\ B == \subst{A'}{X}{B} : \subst{K}{X}{B}$. Finally,
      \ref{lemma:rel-subst} and transitivity of type equality gives
      $\langle \Gamma, A\ B \rangle \in \KDenotE{\subst{K}{X}{B}}$.

    \item Case \textsc{K-Abs-\#}:

      The relevant premise is $\Gamma, X:J \vdash A : K$. Note that we cannot
      use this as an inductive hypothesis directly, as it is not tight.
      Instead, let $B$ be such that $\langle \Gamma, B \rangle \in \KDenotE{J}$.
      By Lemma \ref{lemma:inv}, we have $\Gamma \vdash_\# B : J$ which can be
      weakened back to full DOT typing, giving $\Gamma \vdash B : J$. By
      narrowing on the variable $X$, we have $\Gamma, X:\Sing{B}{J} \vdash
      A : K$. Note that $\Gamma, X:\Sing{B}{J}$ is a store. Stores are inert,
      so we can convert back to tight typing to get $\Gamma, X:\Sing{B}{J}
      \vdash_\# A : K$. \emph{Now} we cite the inductive hypothesis to get
      $\langle (\Gamma, X:\Sing{B}{J}), A \rangle \in \KDenotE{K}$, as
      desired. This is well-founded because $K$ is syntactically smaller than
      $\KDepArr{X}{J}{K}$.

    \item Case \textsc{K-Sing-\#}:

      The premise is $\Gamma \vdash_\# A : \interval{S}{U}$. By the inductive
      hypothesis, we get a whnf $\tau$ such that $\Gamma \vdash_\# A == \tau :
      \interval{S}{U}$. \textcolor{red}{The definition of the relation requires
      showing $\Gamma \vdash_\# A \le \tau \le A: \TyKd$. This is not obviously
      true, but I think it is because proper types can not bind any "new"
      variables that aren't already reflected in $\Gamma$.}
  \end{itemize}
\end{proof}

\newpage
\appendix

\section{$\DOTw$ Full rules}\label{appendix:dotw-rules}

\begin{figure}[ht]
  \begin{mathpar}
    \inferrule*{\phantom{}}{\isctx{\varnothing}} \and
    \inferrule*{\isctx{\Gamma}\and\Gamma \vdash \iskd{K}}
      {\isctx{\Gamma, X:K}} \and
    \inferrule*{\isctx{\Gamma}\and\Gamma \vdash A : \TyKd}
      {\isctx{\Gamma, x:A}} \and
  \end{mathpar}
  \caption{Context formation}
\end{figure}

\begin{figure}[ht]
  \begin{mathpar}
    \inferrule*[right=Wf-Intv]
      {\Gamma \vdash S : \TyKd \and \Gamma \vdash U : \TyKd}
      {\Gamma \vdash \iskd{\interval{S}{U}}} \and
    \inferrule*[right=Wf-DArr]
      {\Gamma \vdash \iskd{J} \and \Gamma, X:J \vdash \iskd{K}}
      {\Gamma \vdash \iskd{\KDepArr{X}{J}{K}}}
  \end{mathpar}
  \caption{Kind formation}
\end{figure}

\begin{figure}[ht]
  \begin{mathpar}
    \inferrule*[right=SK-Intv]
      {\Gamma \vdash S_2 \le S_1:\TyKd \and \Gamma \vdash U_1 \le U_2:\TyKd}
      {\Gamma \vdash \interval{S_1}{U_1} \le \interval{S_2}{U_2}} \and
    \inferrule*[right=SK-DArr]
      {\Gamma \vdash \iskd{\KDepArr{X}{J_1}{K_1}} \and
       \Gamma \vdash J_2 \le J_1 \and
       \Gamma, X:J_2 \vdash K_1 \le K_2}
      {\Gamma \vdash \KDepArr{X}{J_1}{K_1} \le \KDepArr{X}{J_2}{K_2}}
  \end{mathpar}
  \caption{Subkinding}
\end{figure}

\begin{figure}[ht]
  \begin{mathpar}
    \inferrule*[right=K-Var]{\isctx{\Gamma, X:K}}{\Gamma, X:K \vdash X:K} \and
    \inferrule*[right=K-Top]{\phantom{}}{\Gamma \vdash \top : \TyKd} \and
    \inferrule*[right=K-Bot]{\phantom{}}{\Gamma \vdash \bot : \TyKd} \and
    \inferrule*[right=K-Sing]{\Gamma \vdash A : \interval{S}{U}}
      {\Gamma \vdash A : \interval{A}{A}} \and
    \inferrule*[right=K-Arr]
      {\Gamma \vdash A : \TyKd \and \Gamma, x : A \vdash B : \TyKd}
      {\Gamma \vdash \TDepArr{x}{A}{B} : \TyKd} \and
    \inferrule*[right=K-Abs]
      {\Gamma \vdash \iskd{J} \and \Gamma, X:J \vdash A : K \and
       \Gamma, X:J \vdash \iskd{K}}
      {\Gamma \vdash \lambda(X:J).A : \KDepArr{X}{J}{K}} \and
    \inferrule*[right=K-App]
      {\Gamma \vdash A : \KDepArr{X}{J}{K} \and \Gamma \vdash B : J \and
       \Gamma, X:J \vdash \iskd{K} \and \Gamma \vdash \iskd{\subst{K}{B}{X}}}
      {\Gamma \vdash A\ B : \subst{K}{B}{X}} \and
    \inferrule*[right=K-Intersect]
      {\Gamma \vdash A : \interval{S_1}{U_1} \and
       \Gamma \vdash B : \interval{S_2}{U_2}}
      {\Gamma \vdash A \land B : \interval{S_1 \lor S_2}{U_1 \land U_2}} \and
    \inferrule*[right=K-Field]
      {\Gamma \vdash A : \interval{S}{U}}
      {\Gamma \vdash \objtyp{val}{\ell}{A} : *} \and
    \inferrule*[right=K-Typ]
      {\Gamma \vdash \iskd{K}}
      {\Gamma \vdash \objtyp{type}{M}{K} : *} \and
    \inferrule*[right=K-Typ-Mem]
      {\Gamma \vdash x : \objtyp{type}{M}{K}}
      {\Gamma \vdash x.M : K} \and
    \inferrule*[right=K-Rec]
      {\Gamma, x : \tau \vdash \tau : K}
      {\Gamma \vdash \mu(x . \tau) : K} \and
    \inferrule*[right=K-Sub]
      {\Gamma \vdash A:J \and \Gamma \vdash J \le K}{\Gamma \vdash A:K}
  \end{mathpar}
  \caption{Kind assignment}
\end{figure}
\clearpage

Note that \textsc{K-Intersect} rules refers to the union type $S_1 \lor S_2$,
despite no such construct being present in the language as a whole. I am
currently investigating whether the explicit addition of this construct is
necessary.

\begin{figure}[ht]
  \begin{mathpar}
    \inferrule*[right=ST-refl]{\Gamma \vdash A : K}
      {\Gamma \vdash A \le A : K} \and
    \inferrule*[right=ST-trans]
      {\Gamma \vdash A \le B : K \and
       \Gamma \vdash B \le C : K}
      {\Gamma \vdash A \le C : K} \and
    \inferrule*[right=ST-top]{\Gamma \vdash A : \interval{S}{U}}
      {\Gamma \vdash A \le \top : \TyKd} \and
    \inferrule*[right=ST-bot]{\Gamma \vdash A : \interval{S}{U}}
      {\Gamma \vdash \bot \le A : \TyKd} \and
    \inferrule*[right=ST-and-$\ell_1$]
      {\Gamma \vdash A \land B : K}{\Gamma \vdash A \land B \le A : K} \and
    \inferrule*[right=ST-and-$\ell_2$]
      {\Gamma \vdash A \land B : K}{\Gamma \vdash A \land B \le B : K} \and
    \inferrule*[right=ST-and-r]
      {\Gamma \vdash S \le A : K \and \Gamma \vdash S \le B : K}
      {\Gamma \vdash S \le A \land B : K} \and
    \inferrule*[right=ST-field]{\Gamma \vdash A \le B : \TyKd}
      {\Gamma \vdash \objtyp{val}{\ell}{A} \le \objtyp{val}{\ell}{B} : \TyKd}
      \and
    \inferrule*[right=ST-typ]
      {\Gamma \vdash J \le K}
      {\Gamma \vdash \objtyp{type}{M}{J} \le \objtyp{type}{M}{K} : \TyKd}\and
    \inferrule*[right=ST-$\beta_1$]
      {\Gamma Z : J \vdash A : K \and
       \Gamma \vdash B : J}
      {\Gamma \vdash (\lambda (Z: J).A)\ B \le \subst{A}{Z}{B} : \subst{K}{Z}{B}}
      \and
    \inferrule*[right=ST-$\beta_2$]
      {\Gamma Z : J \vdash A : K \and
       \Gamma \vdash B : J}
      {\Gamma \vdash \subst{A}{Z}{B} \le (\lambda (Z: J).A)\ B : \subst{K}{Z}{B}}
      \and
    \inferrule*[right=ST-bnd$_1$]
      {\Gamma \vdash A : \interval{S}{U}}
      {\Gamma \vdash S \le A : *}
      \and
    \inferrule*[right=ST-bnd$_2$]
      {\Gamma \vdash A : \interval{S}{U}}
      {\Gamma \vdash A \le U : *}
  \end{mathpar}
  \caption{Subtyping}
\end{figure}

\begin{figure}[ht]
  \begin{mathpar}
    \inferrule*[right=Eq]
      {\Gamma \vdash A \le B : K \and \Gamma \vdash B \le A : K}
      {\Gamma \vdash A = B : K}
  \end{mathpar}
  \caption{Type equality}
\end{figure}

\begin{figure}[ht]
  \begin{mathpar}
    \inferrule*[right=Ty-Var]
      {\isctx{\Gamma, x: \tau}}
      {\Gamma, x:\tau \vdash x: \tau} \and
    \inferrule*[right=Ty-Let]
      {\Gamma \vdash e_1: \tau \and \Gamma, x:\tau \vdash e_2 : \rho \and x \not\in fv(\rho)}
      {\Gamma \vdash \termlet{x}{e_1}{e_2} : \rho} \and
    \inferrule*[right=Ty-Fun-I]
      {\Gamma, x: \tau \vdash e: \rho}
      {\Gamma \vdash \lambda(x:\tau).e : \TDepArr{x}{\tau}{\rho}} \and
    \inferrule*[right=Ty-Fun-E]
      {\Gamma \vdash x : \TDepArr{z}{\tau}{\rho} \and \Gamma \vdash y : \tau}
      {\Gamma \vdash x\ y : \subst{\rho}{z}{y}} \and
    \inferrule*[right=Ty-$\mu$-I]
      {\Gamma \vdash x: \tau}
      {\Gamma \vdash x: \mu(x: \tau)} \and
    \inferrule*[right=Ty-$\mu$-E]
      {\Gamma \vdash x: \mu(z: \tau)}
      {\Gamma \vdash x: \subst{\tau}{z}{x}} \and
    \inferrule*[right=Ty-Rec-I]
      {\Gamma, x: \tau \vdash d: \tau}
      {\Gamma \vdash \nu(x: \tau)d : \mu(x:\tau)} \and
    \inferrule*[right=Ty-Rec-E]
      {\Gamma, x: \objtyp{val}{\ell}{\tau}}
      {\Gamma \vdash x.\ell : \tau} \and
    \inferrule*[right=Ty-And-I]
      {\Gamma \vdash x: \tau_1 \and \Gamma \vdash x: \tau_2}
      {\Gamma \vdash x: \tau_1 \land \tau_2} \and
    \inferrule*[right=Ty-Sub]
      {\Gamma \vdash e: \tau_1 \and \Gamma \vdash \tau_1 \le \tau_2 : *}
      {\Gamma \vdash e: \tau_2} \and
    \inferrule*[right=Ty-Def-Trm]
      {\Gamma \vdash e: \rho}
      {\Gamma \vdash \objval{val}{\ell}{e} : \objtyp{val}{\ell}{\rho}} \and
    \inferrule*[right=Ty-Def-Typ]
      {\Gamma \vdash \tau : K}
      {\Gamma \vdash \objval{type}{M}{A} : \objtyp{type}{M}{\Sing{A}{K}}}
  \end{mathpar}
  \caption{Type assignment}
\end{figure}

\section{Tight typing rules}\label{appendix:tight-rules}

In most cases, tight typing is merely forwarded to the premises. In any rule
that extends the context with possibly-untrusted bounds, tight typing reverts
to general typing.

\begin{figure}[ht]
  \begin{mathpar}
    \inferrule*{\phantom{}}{\istctx{\varnothing}} \and
    \inferrule*{\istctx{\Gamma}\and\Gamma \vdash_\# \iskd{K}}
      {\isctx{\Gamma, X:K}} \and
    \inferrule*{\istctx{\Gamma}\and\Gamma \vdash_\# A : \TyKd}
      {\isctx{\Gamma, x:A}} \and
  \end{mathpar}
  \caption{Context formation}
\end{figure}

\begin{figure}[ht]
  \begin{mathpar}
    \inferrule*[right=Wf-Int-\#]
      {\Gamma \vdash_\# S : \TyKd \and \Gamma \vdash_\# U : \TyKd}
      {\Gamma \vdash_\# \iskd{\interval{S}{U}}} \and
    \inferrule*[right=Wf-DArr-\#]
      {\Gamma \vdash_\# \iskd{J} \and \Gamma, X:J \vdash_\# \iskd{K}}
      {\Gamma \vdash_\# \iskd{\KDepArr{X}{J}{K}}}
  \end{mathpar}
  \caption{Kind formation}
\end{figure}

\begin{figure}[ht]
  \begin{mathpar}
    \inferrule*[right=SK-Intv-\#]
      {\Gamma \vdash_\# S_2 \le S_1:\TyKd \and \Gamma \vdash_\# U_1 \le U_2:\TyKd}
      {\Gamma \vdash_\# \interval{S_1}{U_1} \le \interval{S_2}{U_2}} \and
    \inferrule*[right=SK-DArr-\#]
      {\Gamma \vdash_\# \iskd{\KDepArr{X}{J_1}{K_1}} \and
       \Gamma \vdash_\# J_2 \le J_1 \and
       \Gamma, X:J_2 \vdash K_1 \le K_2}
      {\Gamma \vdash_\# \KDepArr{X}{J_1}{K_1} \le \KDepArr{X}{J_2}{K_2}}
  \end{mathpar}
  \caption{Subkinding}
\end{figure}

\begin{figure}[ht]
  \begin{mathpar}
    \inferrule*[right=K-Var-\#]{\istctx{\Gamma, X:K}}{\Gamma, X:K \vdash_\# X:K} \and
    \inferrule*[right=K-Top-\#]{\phantom{}}{\Gamma \vdash_\# \top : \TyKd} \and
    \inferrule*[right=K-Bot-\#]{\phantom{}}{\Gamma \vdash_\# \bot : \TyKd} \and
    \inferrule*[right=K-Sing-\#]{\Gamma \vdash_\# A : \interval{S}{U}}
      {\Gamma \vdash_\# A : \interval{A}{A}} \and
    \inferrule*[right=K-Arr-\#]
      {\Gamma \vdash_\# A : \TyKd \and \Gamma, x : A \vdash B : \TyKd}
      {\Gamma \vdash_\# \TDepArr{x}{A}{B} : \TyKd} \and
    \inferrule*[right=K-Abs-\#]
      {\Gamma \vdash_\# \iskd{J} \and \Gamma, X:J \vdash A : K \and
       \Gamma, X:J \vdash_\# \iskd{K}}
      {\Gamma \vdash_\# \lambda(X:J).A : \KDepArr{X}{J}{K}} \and
    \inferrule*[right=K-App-\#]
      {\Gamma \vdash_\# A : \KDepArr{X}{J}{K} \and \Gamma \vdash_\# B : J \and
       \Gamma, X:J \vdash \iskd{K} \and \Gamma \vdash_\# \iskd{\subst{K}{B}{X}}}
      {\Gamma \vdash_\# A\ B : \subst{K}{B}{X}} \and
    \inferrule*[right=K-Intersect-\#]
      {\Gamma \vdash_\# A : \interval{S_1}{U_1} \and
       \Gamma \vdash_\# B : \interval{S_2}{U_2}}
      {\Gamma \vdash_\# A \land B : \interval{S_1 \lor S_2}{U_1 \land U_2}} \and
    \inferrule*[right=K-Field-\#]
      {\Gamma \vdash_\# A : \interval{S}{U}}
      {\Gamma \vdash_\# \objtyp{val}{\ell}{A} : *} \and
    \inferrule*[right=K-Typ-\#]
      {\Gamma \vdash_\# \iskd{K}}
      {\Gamma \vdash_\# \objtyp{type}{M}{K} : *} \and
    \inferrule*[right=K-Typ-Mem-\#]
      {\Gamma \vdash_! x : \objtyp{type}{M}{Sing{A}{K}}}
      {\Gamma \vdash_\# x.M : Sing{A}{K}} \and
    \inferrule*[right=K-Rec-\#]
      {\Gamma, x : \tau \vdash \tau : K}
      {\Gamma \vdash_\# \mu(x . \tau) : K} \and
    \inferrule*[right=K-Sub-\#]
      {\Gamma \vdash_\# A:J \and \Gamma \vdash_\# J \le K}{\Gamma \vdash_\# A:K}
  \end{mathpar}
  \caption{Kind assignment}
\end{figure}

\begin{figure}[ht]
  \begin{mathpar}
    \inferrule*[right=ST-refl-\#]{\Gamma \vdash_\# A : K}
      {\Gamma \vdash_\# A \le A : K} \and
    \inferrule*[right=ST-trans-\#]
      {\Gamma \vdash_\# A \le B : K \and
       \Gamma \vdash_\# B \le C : K}
      {\Gamma \vdash_\# A \le C : K} \and
    \inferrule*[right=ST-top-\#]{\Gamma \vdash_\# A : \interval{S}{U}}
      {\Gamma \vdash_\# A \le \top : \TyKd} \and
    \inferrule*[right=ST-bot-\#]{\Gamma \vdash_\# A : \interval{S}{U}}
      {\Gamma \vdash_\# \bot \le A : \TyKd} \and
    \inferrule*[right=ST-and-$\ell_1$-\#]
      {\Gamma \vdash_\# A \land B : K}{\Gamma \vdash_\# A \land B \le A : K} \and
    \inferrule*[right=ST-and-$\ell_2$-\#]
      {\Gamma \vdash_\# A \land B : K}{\Gamma \vdash_\# A \land B \le B : K} \and
    \inferrule*[right=ST-and-r-\#]
      {\Gamma \vdash_\# S \le A : K \and \Gamma \vdash_\# S \le B : K}
      {\Gamma \vdash_\# S \le A \land B : K} \and
    \inferrule*[right=ST-field-\#]{\Gamma \vdash_\# A \le B : \TyKd}
      {\Gamma \vdash_\# \objtyp{val}{\ell}{A} \le \objtyp{val}{\ell}{B} : \TyKd}
      \and
    \inferrule*[right=ST-typ-\#]
      {\Gamma \vdash_\# J \le K}
      {\Gamma \vdash_\# \objtyp{type}{M}{J} \le \objtyp{type}{M}{K} : \TyKd}\and
    \inferrule*[right=ST-$\beta_1$-\#]
      {\Gamma \vdash_\# B : J \and
       \Gamma, Z: \Sing{B}{J} \vdash_\# A : K}
      {\Gamma \vdash_\# (\lambda (Z: J).A)\ B \le \subst{A}{Z}{B} : \subst{K}{Z}{B}}
      \and
    \inferrule*[right=ST-$\beta_2$-\#]
      {\Gamma \vdash_\# B : J \and
       \Gamma, Z: \Sing{B}{J} \vdash_\# A : K}
      {\Gamma \vdash_\# \subst{A}{Z}{B} \le (\lambda (Z: J).A)\ B : \subst{K}{Z}{B}}
    \inferrule*[right=ST-bnd$_1$-\#]
      {\Gamma \vdash_\# A : \interval{S}{U}}
      {\Gamma \vdash_\# S \le A : *}
      \and
    \inferrule*[right=ST-bnd$_2$-\#]
      {\Gamma \vdash_\# A : \interval{S}{U}}
      {\Gamma \vdash_\# A \le U : *}
  \end{mathpar}
  \caption{Subtyping}
\end{figure}

\begin{figure}[ht]
  \begin{mathpar}
    \inferrule*[right=Eq-\#]
      {\Gamma \vdash_\# A \le B : K \and \Gamma \vdash_\# B \le A : K}
      {\Gamma \vdash_\# A = B : K}
  \end{mathpar}
  \caption{Type equality}
\end{figure}

\begin{figure}[ht]
  \begin{mathpar}
    \inferrule*[right=Ty-Var-\#]
      {\istctx{\Gamma, x: \tau}}
      {\Gamma, x:\tau \vdash_\# x: \tau} \and
    \inferrule*[right=Ty-Let-\#]
      {\Gamma \vdash_\# e_1: \tau \and \Gamma, x:\tau \vdash e_2: \rho \and x \not\in fv(\rho)}
      {\Gamma \vdash_\# \termlet{x}{e_1}{e_2} : \rho} \and
    \inferrule*[right=Ty-Fun-I-\#]
      {\Gamma, x: \tau \vdash e: \rho}
      {\Gamma \vdash_\# \lambda(x:\tau).e : \TDepArr{x}{\tau}{\rho}} \and
    \inferrule*[right=Ty-Fun-E-\#]
      {\Gamma \vdash_\# x : \TDepArr{z}{\tau}{\rho} \and \Gamma \vdash_\# y : \tau}
      {\Gamma \vdash_\# x\ y : \subst{\rho}{z}{y}} \and
    \inferrule*[right=Ty-$\mu$-I-\#]
      {\Gamma \vdash_\# x: \tau}
      {\Gamma \vdash_\# x: \mu(x: \tau)} \and
    \inferrule*[right=Ty-$\mu$-E-\#]
      {\Gamma \vdash_\# x: \mu(z: \tau)}
      {\Gamma \vdash_\# x: \subst{\tau}{z}{x}} \and
    \inferrule*[right=Ty-Rec-I-\#]
      {\Gamma, x: \tau \vdash d: \tau}
      {\Gamma \vdash_\# \nu(x: \tau)d : \mu(x:\tau)} \and
    \inferrule*[right=Ty-Rec-E-\#]
      {\Gamma, x: \objtyp{val}{\ell}{\tau}}
      {\Gamma \vdash_\# x.\ell : \tau} \and
    \inferrule*[right=Ty-And-I-\#]
      {\Gamma \vdash_\# x: \tau_1 \and \Gamma \vdash_\# x: \tau_2}
      {\Gamma \vdash_\# x: \tau_1 \land \tau_2} \and
    \inferrule*[right=Ty-Sub-\#]
      {\Gamma \vdash_\# e: \tau_1 \and \Gamma \vdash_\# \tau_1 \le \tau_2 : *}
      {\Gamma \vdash_\# e: \tau_2} \and
    \inferrule*[right=Ty-Def-Trm-\#]
      {\Gamma \vdash_\# e: \rho}
      {\Gamma \vdash_\# \objval{val}{\ell}{e} : \objtyp{val}{\ell}{\rho}} \and
    \inferrule*[right=Ty-Def-Typ-\#]
      {\Gamma \vdash_\# \tau : K}
      {\Gamma \vdash_\# \objval{type}{M}{A} : \objtyp{type}{M}{Sing{A}{K}}}
  \end{mathpar}
  \caption{Type assignment}
\end{figure}

\begin{figure}[ht]
  \begin{mathpar}
    \inferrule*[right=Var-!]{\phantom{}}{\Gamma, x:\tau \vdash_! x:\tau}\and
    \inferrule*[right=Rec-E-!]{\Gamma \vdash_! x: \mu(z:\tau)}
      {\Gamma \vdash_! x: \subst{\tau}{z}{x}} \and
    \inferrule*[right=And$_1$-E-!]
      {\Gamma \vdash_! x: \tau_1 \land \tau_2}
      {\Gamma \vdash_! x: \tau_1}\and
    \inferrule*[right=And$_2$-E-!]
      {\Gamma \vdash_! x: \tau_1 \land \tau_2}
      {\Gamma \vdash_! x: \tau_2}\and
    \inferrule*[right=Fun-I-!]
      {\Gamma, x: \tau \vdash e: \rho \and x \not\in fv(\tau)}
      {\Gamma \vdash_! \lambda(x:\tau).e : \TDepArr{x}{\tau}{\rho}} \and
    \inferrule*[right=Record-I-!]
      {\Gamma, x: \tau \vdash d: \tau}
      {\Gamma \vdash_! \nu(x:\tau)d : \mu(x:\tau)}
  \end{mathpar}
  \caption{Precise value and variable typing}
\end{figure}

\section{Invertible Typing}\label{appendix:invertible}

\begin{figure}[ht]
  \begin{mathpar}
    \inferrule*[right=Var-\#\#]{\Gamma \vdash_! x: \tau}{\Gamma \vdash_{\#\#} x: \tau}\and
    \inferrule*[right=Val-\#\#]
      { \Gamma \invdash x: \objtyp{val}{\ell}{\tau} \and
        \Gamma \vdash_\# \tau \le \rho : *}
      { \Gamma \invdash x: \objtyp{val}{\ell}{\rho} } \and
    \inferrule*[right=Type-\#\#]
      { \Gamma \invdash x: \objtyp{type}{M}{J} \and
        \Gamma \vdash_\# J \le K}
      { \Gamma \invdash x: \objtyp{type}{M}{K} } \and
    \inferrule*[right=Fun-\#\#]
      { \Gamma \invdash x: \TDepArr{z}{S}{T} \and
        \Gamma \vdash_\# S' \le S : J \and
        \Gamma, z: S' \vdash_\# T \le T' : K }
      { \Gamma \invdash x : \TDepArr{z}{S'}{T'} } \and
    \inferrule*[right=Intersect-\#\#]
      { \Gamma \invdash x: A \and \Gamma \invdash x: B }
      { \Gamma \invdash x: A \land B } \and
    \inferrule*[right=Sel-\#\#]
      { \Gamma \invdash x: A \and
        \Gamma \vdash_! z : \objtyp{type}{M}{\interval{A}{A}} }
      { \Gamma \invdash x : z.M } \and
    \inferrule*[right=Rec-I-\#\#]
      { \Gamma \invdash x: \tau }
      { \Gamma \invdash x: \mu(x: \tau) } \and
    \inferrule*[right=Top-\#\#]
      { \Gamma \invdash x: \tau }
      { \Gamma \invdash x: \top }
  \end{mathpar}
  \caption{Invertible value and variable typing}
\end{figure}

\section{Auxiliary Lemmas}

\begin{lemma}[Tight to invertible typing]
  For inert contexts $\Gamma$, $\Gamma \vdash_\# x: \tau$ implies
  $\Gamma \invdash x: \tau$, and for all values $v$, $\Gamma \vdash_\# v: \tau$
  implies $\Gamma \invdash v: \tau$.
\end{lemma}
\begin{proof}
  By straightforward induction on $\Gamma \vdash_\# x: \tau$ and $\Gamma
  \vdash_\# v: \tau$. \textcolor{red}{This is formalized in Agda.}
\end{proof}

\section{$\DOTw{}$ Operational Semantics}

\begin{figure}[ht]
  \begin{mathpar}
    \inferrule*[right=Term]
      {t \mapsto t'}
      {E[t] \mapsto E[t']} \\
    \inferrule*[right=Apply]
      {v = \lambda(z: \tau).t}
      {\termlet{x}{v}{E[x\ y]} \mapsto \termlet{x}{v}{E[\subst{t}{y}{z}]}} \and
    \inferrule*[right=Project]
      {v = \nu(x:\tau)...\objval{val}{\ell}{t}}
      {\termlet{x}{v}{E[x.\ell]} \mapsto \termlet{x}{v}{E[t]}} \\
    \inferrule*[right=Let-Var]
      {\strut}
      {\termlet{x}{y}{t} \mapsto \subst{t}{y}{x}} \and
    \inferrule*[right=Let-Let]
      {\strut}
      {\termlet{x}{(\termlet{y}{e}{t'})}{t} \mapsto \termlet{y}{e}{\termlet{x}{t'}{t}}}
  \end{mathpar}
  \caption{$\DOTw$ Operational Semantics (\citet{amin2016})}
\end{figure}

\begin{figure}[ht]
  \begin{mathpar}
    \inferrule*[right=Apply]
      {E\text{ contains the binding }\envlet{x}{\lambda(z:\tau).t}}
      {E[x\ y] \mapsto E[\subst{t}{y}{z}]} \and
    \inferrule*[right=Project]
      {E\text{ contains the binding }\envlet{x}{\nu(x: \tau)...\objval{val}{\ell}{t}}}
      {E[x.\ell] \mapsto E[t]} \\
    \inferrule*[right=Let-Var]
      {\strut}
      {E[\termlet{x}{[y]}{t}] \mapsto E[\subst{t}{y}{x}]} \and
    \inferrule*[right=Let-Let]
      {\strut}
      {E[\termlet{x}{[\termlet{y}{e}{t'}]}{t}] \mapsto E[\termlet{y}{e}{\termlet{x}{t'}{t}}]}
  \end{mathpar}
  \caption{$\DOTw$ Operational Semantics with inlined \textsc{Term} (\citet{rapoport2017})}
\end{figure}

\end{document}
