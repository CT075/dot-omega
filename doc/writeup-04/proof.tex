\documentclass[a4paper, 10pt]{article}
\usepackage{amssymb, amsmath, amsthm}
\usepackage[backend=bibtex,natbib=true]{biblatex}
\usepackage{enumitem}
\usepackage{mathrsfs}
\usepackage{parskip}
\usepackage{mathpartir}
\usepackage{bussproofs}
\usepackage{stmaryrd}
\usepackage{appendix}
\usepackage{datetime}
\usepackage{tabularx}
\usepackage{xcolor}
\usepackage{pifont}
\usepackage[iso, english]{isodate}

\newcommand{\xmark}{\ding{55}}

\newcommand{\DOTw}{\ensuremath{DOT^\omega}}
\newcommand{\Fwint}{\ensuremath{F^\omega_{..}}}
\newcommand{\Dsub}{\ensuremath{D_{<:}}}
\newcommand{\interval}[3][]{#2 .._{#1} #3}
\newcommand{\isctx}[1]{#1\ \texttt{ctx}}
\newcommand{\istctx}[1]{#1\ \texttt{ctx}_\#}
\newcommand{\iskd}[1]{#1\ \texttt{kd}}
\newcommand{\TyKd}{*}
\newcommand{\KDepArr}[3]{\Pi(#1:#2).#3}
\newcommand{\TDepArr}[3]{(#1:#2) \rightarrow #3}
\newcommand{\subst}[3]{#1[#2/#3]}
\newcommand{\objtyp}[3]{\{ \textbf{#1}\ #2 : #3 \}}
\newcommand{\objval}[3]{\{ \textbf{#1}\ #2 = #3 \}}
\newcommand{\termlet}[3]{\text{let }#1 = #2\text{ in }#3}

\newtheorem{theorem}{Theorem}
\newtheorem{defn}{Definition}
\newtheorem{lemma}{Lemma}
\newtheorem{property}{Property}

\bibliography{paper}

\setlength{\parindent}{0cm}

\title{Soundness for \DOTw{} (update \today)}
\author{Cameron Wong}

\begin{document}
\maketitle

\setlength{\parskip}{\baselineskip}

\iffalse
\section{Background}

The key difficulty in introducing higher kinded types to DOT is, like with any
DOT soundness proof, bad bounds. Particularly, in an open context, neither
types nor terms satisfy subject reduction, making it unsound to perform
evaluation in the presence of arbitrary subtyping lattices. This phenomenon
is well-documented by \citet{amin2016}, \citet{rapoport2017},
\citet{stucki2017} and others besides.

The usual approach to dealing with this problem is to consider soundness only
in closed terms, or in limited contexts where all bounds are known to be good.
This notion was originally formalized by \citet{amin2016} as \emph{tight
typing}. \citet{rapoport2017} then demonstrated how tight typing can be used to
dramatically simplify the proof of soundness for DOT. This was done by defining
\emph{inert typing contexts}, in which tight typing and general typing for DOT
are equivalent. This allowed them to prove properties of the overall DOT type
system by first dropping into tight typing, then using simple, intuitive
reasoning over the tightly-typed system. In the words of \citet{rapoport2017},
``To reconcile a subtyping lattice with a sound language, we only need to force
the programmer to provide evidence that the custom lattice does not violate any
familiar assumptions... This evidence takes the form of an argument value that
must be passed to the lambda before the body that uses the custom type lattice
can be allowed to execute.''.

A major difficulty in extending this recipe to cover higher-kinded types is
that, when types can contain computation themselves, some $\beta$-reduction
\emph{has} to be performed in the typing rules (e.g., to compare types during
typechecking), which is similarly unsound in the presence of bad bounds.
This, too, can be resolved by making the following observation:

\begin{property}
  If $\varnothing \vdash e : \tau$, then $\varnothing \vdash \tau : *$.
\end{property}

That is, any closed expression $e$ must be assigned a proper type (as opposed
to a type lambda) if it has any type at all. This means that, even if $\tau$ is
a complex type expression containing many type functions, each of those
functions must ultimately be applied to another type expression witnessing the
validity of any bounds introduced. We will formally define this property in
Section \ref{sec:tight-typing}.
\fi

\section{$\DOTw$ Overview}

\subsection{Syntax}

\begin{figure}[ht]
  \centering
  \begin{tabularx}{\linewidth}{XXXX}
    $x,y,z$... & \textbf{Term Variable} & $X,Y,Z$... & \textbf{Type Variable}\\
    $\ell$ & \textbf{Value Label} & $M$ & \textbf{Type Label} \\
  \end{tabularx}
  \hfill \\
  \hfill \\
  \begin{tabularx}{\linewidth}{XlX}
    $e,t ::=$ &
    $x \mid v \mid x.\ell \mid x\ y \mid \termlet{x}{e}{t}$
      & \textbf{Term} \\
    $v ::=$ & $\lambda(x:\tau).e \mid \nu(x:\tau).d$ & \textbf{Value} \\
    $d ::=$ & $\objval{val}{\ell}{e} \mid \objval{type}{M}{A} \mid
      d_1 \land d_2$ & \textbf{Definition} \\
    $A,B,C ::=$ &
    $X \mid \lambda(X:K).A \mid x.M \mid A\ B$ & \textbf{Type} \\
    $(\tau, \rho, S, U, T ::=)$ &
      $\mid \top \mid \bot \mid \TDepArr{x}{\tau}{\rho}
      \mid \tau \land \rho \mid \mu(x.\tau)$ & (Proper types) \\
    & $\mid \objtyp{val}{\ell}{\tau} \mid \objtyp{type}{M}{K}$ & \\
    $J,K ::=$ & $\interval{S}{U} \mid \KDepArr{X}{J}{K}$ & \textbf{Kind} \\
  \end{tabularx}
  \caption{Abstract syntax of $\DOTw{}$}
\end{figure}

We also use $*$ as shorthand for $\interval{\bot}{\top}$.

Following \citet{stucki2017}, we represent type bounds at the kind level, with
$\Gamma \vdash \tau : \interval{S}{U}$ meaning $S \le \tau$ and $\tau \le U$.
Type members are then assigned a kind, rather than marking bounds
syntactically.

\subsection{Typing and Kinding}

\begin{figure}[ht]
  \textbf{Kind Assignment}
  \begin{mathpar}
    \inferrule*[right=K-Sing]{\Gamma \vdash A : \interval{S}{U}}
      {\Gamma \vdash A : \interval{A}{A}} \and
    \inferrule*[right=K-App]
      {\Gamma \vdash A : \KDepArr{X}{J}{K} \and \Gamma \vdash B : J \and
       \Gamma, X:J \vdash \iskd{K} \and \Gamma \vdash \iskd{\subst{K}{B}{X}}}
      {\Gamma \vdash A\ B : \subst{K}{B}{X}} \and
    \inferrule*[right=K-Typ]
      {\Gamma \vdash \iskd{K}}
      {\Gamma \vdash \objtyp{type}{M}{K} : *} \and
    \inferrule*[right=K-Field]
      {\Gamma \vdash A : \interval{S}{U}}
      {\Gamma \vdash \objtyp{val}{\ell}{A} : *} \and
    \inferrule*[right=K-Typ-Mem]
      {\Gamma \vdash x : \objtyp{type}{M}{K}}
      {\Gamma \vdash x.M : K}
  \end{mathpar}
  \textbf{Subtyping}
  \begin{mathpar}
    \inferrule*[right=ST-field]{\Gamma \vdash A \le B : \TyKd}
      {\Gamma \vdash \objtyp{val}{\ell}{A} \le \objtyp{val}{\ell}{B} : \TyKd}
      \and
    \inferrule*[right=ST-typ]
      {\Gamma \vdash J \le K}
      {\Gamma \vdash \objtyp{type}{M}{J} \le \objtyp{type}{M}{K} : \TyKd}\and
    \inferrule*[right=ST-$\beta_1$]
      {\Gamma Z : J \vdash A : K \and
       \Gamma \vdash B : J}
      {\Gamma \vdash (\lambda (Z: J).A)\ B \le \subst{A}{Z}{B} : \subst{K}{Z}{B}}
      \and
    \inferrule*[right=ST-$\beta_2$]
      {\Gamma Z : J \vdash A : K \and
       \Gamma \vdash B : J}
      {\Gamma \vdash \subst{A}{Z}{B} \le (\lambda (Z: J).A)\ B : \subst{K}{Z}{B}}
  \end{mathpar}
  \caption{Kinding and subtyping in $\DOTw{}$ (selected rules)}
\end{figure}

The full declarative typing system can be found in Appendix
\ref{appendix:dotw-rules}. Most rules are lifted directly from
\citet{stucki2017}, adding back recursive types, type members and
intersection types. We also do not feature an $\eta$ rule, but may re-add it if
necessary.

Notice that, instead of the rules \textsc{<:-Sel} and \textsc{Sel-<:} relating
a type member to its bounds, we instead have one rule \textsc{K-Typ-Mem} giving
each type member its kind.

\section{The Proof}

\subsection{Outline}\label{sec:outline}

Our general proof recipe will largely follow that of \citet{rapoport2017}.
Unlike the first-order system, merely knowing that $\Gamma \vdash v: A$ does
not allow us to deduce the canonical form of the value $v$, as $A$ may be a
complex type expression containing $\beta$-redexes. Nor can we simply normalize
the type expression $A$, as we may be in a crazy context in which subtyping
cannot be trusted and subject reduction does not hold. If we \emph{do} know
that the context is inert, however, then reduction is sound and all is well.

The overall plan of attack, then, will proceed as follows:

\begin{enumerate}
  \item Extend DOT's tight typing rules with corresponding tight \emph{kinding}
    and reduction rules, alongside extending the notion of inert contexts to
    restrict type variables as well.
  \item Show that general typing and kinding implies tight typing and kinding
    in inert contexts.
  \item Prove that, under tight subtyping, reduction of type expressions is
    sound and normalizing.
  \item Finally, use the above to prove the relevant canonical forms lemmas
    necessary to show progress and preservation of $\DOTw$.
\end{enumerate}

None of these are new ideas; the contribution is in applying
\citet{rapoport2017}'s simple proof recipe to address issues raised by
\citet{stucki2017}. Items 1 and 2 have proof-of-concept mechanizations in Agda.

\subsection{Inert Higher-kinded Contexts}

\citet{rapoport2017} defines inert contexts to be contexts assigning inert
types to each variable $x$ contained within, where an inert type $\tau$ is:

\begin{itemize}
  \item A function $\TDepArr{x}{\tau_1}{\tau_2}$, or
  \item A recursive type $\mu(x: \tau)$, where $\tau$ is an intersection of
    field declarations and tight type declarations
    $\objtyp{type}{M}{\interval{S}{S}}$, where all type labels are distinct.
\end{itemize}

To generalize this to higher kinds, then, we must account for kinds assigned to
type variables and kinds of type members $\objtyp{type}{M}{K}$. Fortunately,
this work has already been done for us.

\citet{stucki2021} define \emph{higher-order type intervals} as:
\begin{align*}
  \interval[\interval{A'}{B'}]{A}{B} &::= \interval{A}{B} \\
  \interval[\KDepArr{X}{J}{K}]{A}{B} &::=
    \KDepArr{X}{J}{\interval[K]{A\ X}{B\ X}}
\end{align*}

From here, we can recover \citet{stoneharper2000}'s \emph{generalized
singletons} simply by setting the bounds $A$ and $B$ to be equal, e.g.
\begin{align*}
  S(\tau : \interval{S}{U}) &::= \interval{\tau}{\tau} \\
  S(A : \KDepArr{X}{J}{K}) &::= \KDepArr{X}{J}{S(A\ X : K)}
\end{align*}

This gives the full definition of inert contexts as contexts containing only:
\begin{itemize}
  \item Dependent term functions $\TDepArr{x}{\tau_1}{\tau_2}$,
  \item Recursive types built as the intersection of record fields and type
    declarations $\objtyp{type}{M}{S(A : K)}$ for some $A$ and $K$, and
  \item Singleton kinds $S(A : K)$.
\end{itemize}

Note that we exclude from our definition of singletons intervals that may be
$\beta\eta$ equivalent, such as $\interval{(\lambda (\_:*).\top)\ \top}{\top}$
This is to avoid needing to perform reduction (which, remember, may be
unsound!) when determining whether a kind is indeed a singleton.

\subsection{Tight Reduction}\label{sec:tight-typing}

As nice as singleton kinds are, they are not yet sufficient to tame the
wild world of arbitrary subtyping. For example, the singleton kind
$\KDepArr{X}{\interval{\top}{\bot}}{
  \interval
    {\objtyp{type}{M}{\interval{\top}{\bot}}}
    {\objtyp{type}{M}{\interval{\top}{\bot}}}}$
producing the absurd type $\objtyp{type}{M}{\interval{\top}{\bot}}$ is
inhabited (by a function that assigns $M = X$), so merely restricting type
variables to having singleton kinds is not enough.

Instead, we must use the fact that such a type-level function cannot be called
at all (or, dually, that calling such a function requires witnessing the
goodness of any bounds). Tight typing (\citet{amin2016}) provides just this
functionality for term-level functions, but what about at the type level?

\begin{figure}[ht]
  \begin{mathpar}
    \inferrule*[right=ST-$\beta_1$-\#]
      {\Gamma \vdash_\# B : J \and
       \Gamma, Z: S(B:J) \vdash_\# A : K}
      {\Gamma \vdash_\# (\lambda (Z: J).A)\ B \le \subst{A}{Z}{B} : \subst{K}{Z}{B}}
      \and
    \inferrule*[right=ST-$\beta_2$-\#]
      {\Gamma \vdash_\# B : J \and
       \Gamma, Z: S(B:J) \vdash_\# A : K}
      {\Gamma \vdash_\# \subst{A}{Z}{B} \le (\lambda (Z: J).A)\ B : \subst{K}{Z}{B}}
  \end{mathpar}
  \label{figure::beta-tight}
  \caption{Tight Type Reduction}
\end{figure}

To recover subject reduction, we must amend the $\beta$-reduction rules to
ensure that it does not make use of any strange subtyping relationships
that would not be present otherwise. This is done by type checking the lambda
body under the restricted context $\Gamma, Z: S(B:J)$ rather than
$\Gamma, Z:J$ (fig. \ref{figure::beta-tight}).

\begin{figure}[ht]
  \textbf{Precise Variable Typing}
  \begin{mathpar}
    \inferrule*[right=Var-!]{\phantom{}}{\Gamma, x:\tau \vdash_! x:\tau}\and
    \inferrule*[right=Rec-E-!]{\Gamma \vdash_! x: \mu(z:\tau)}
      {\Gamma \vdash_! x: \subst{\tau}{z}{x}} \and
    \inferrule*[right=And$_1$-E-!]
      {\Gamma \vdash_! x: \tau_1 \land \tau_2}
      {\Gamma \vdash_! x: \tau_1}\and
    \inferrule*[right=And$_2$-E-!]
      {\Gamma \vdash_! x: \tau_1 \land \tau_2}
      {\Gamma \vdash_! x: \tau_2}\and
  \end{mathpar}
  \textbf{Tight Kinding (selected rule)}
  \begin{mathpar}
    \inferrule*[right=K-Typ-Mem-\#]
      {\Gamma \vdash_! x : \objtyp{type}{M}{S(A:K)}}
      {\Gamma \vdash_\# x.M : S(A:K)}
  \end{mathpar}
  \label{figure::member-tight}
  \caption{Adapting precise typing (\citet{amin2016}) to a higher-kinded setting}
\end{figure}

We also amend \textsc{K-Typ-Mem}, generalizing \citet{amin2016}'s tight
type member rule to act on higher-order singletons (fig.
\ref{figure::member-tight}). The full proposed tight typing rules can be found
in Appendix \ref{appendix:tight-rules}.

We now show that this restricted setting is equivalent to the general
typing/kinding rules. We do this by making use of two lemmas:

\begin{lemma}[$\beta$-\# Premise]\label{lemma:beta-premise}
  If $\Gamma$ is inert, then if $\Gamma, X : J \vdash A : K$ and
  $\Gamma \vdash_\# B : J$, then $\Gamma, X : S(B:J) \vdash_\# A : K$
\end{lemma}

\begin{lemma}[\textsc{K-Typ-Mem}-\# Premise]\label{lemma:k-typ-mem-premise}
  If $\Gamma$ is inert, then if $\Gamma \vdash_\# x : \objtyp{type}{M}{K}$,
  then there exists some  $A$ such that $\Gamma \vdash_\# A: K$ and $\Gamma
  \vdash_! x : \objtyp{type}{M}{S(A:K)}$
\end{lemma}

The proof of Lemma \ref{lemma:k-typ-mem-premise} follows largely the same as
\citet{rapoport2017} (the details of defining invertible tight typing for
$\DOTw$ are straightforward and therefore elided).

Unfortunately, Lemma \ref{lemma:beta-premise} cannot be stated entirely in
terms of tight typing. The culprit is the $\Gamma, X:J \vdash A:K$ in the
premise of the \textsc{ST-$\beta$} rules, which cannot be converted to a
tightly-typed equivalent (because $J$ may introduce bad bounds). Instead,
we must prove this lemma mutuall alongside the full equivalence theorem:

\begin{theorem}\label{thm:general-tight}
  If $\Gamma$ is an inert context such that $\Gamma \vdash e : \tau$, then
  $\Gamma \vdash_\# e : \tau$.
\end{theorem}
\begin{proof}
  Following \citet{rapoport2017}, we focus on showing equivalence of individual
  rules rather than the typing/kinding relations as a whole.

  \begin{itemize}
    \item \textsc{K-Typ-Mem} discharges to \textsc{K-Typ-Mem-\#} via Lemma
      \ref{lemma:k-typ-mem-premise}
    \item \textsc{ST-$\beta_1$} and \textsc{ST-$\beta_2$} similarly invoke the
      corresponding tight-typing rules, with premises obtained from Lemma
      \ref{lemma:beta-premise}.
  \end{itemize}
\end{proof}

\begin{proof}[Proof of Lemma \ref{lemma:beta-premise}]
  We elide a proof that $\Gamma \vdash S(B:J) \le J$.

  By narrowing, $\Gamma, X:S(B:J) \vdash A:K$. By Theorem \ref{thm:general-tight},
  $\Gamma, X:S(B:J) \vdash_\# A:K$ as desired.
\end{proof}

Note that, for this mutual proof to be well-founded, the proof of the narrowing
theorem must ensure that the proof of $\Gamma, X:S(B:J) \vdash A:K$ is at most
as large as the proof of $\Gamma, X:J \vdash A:K$, which thankfully, it does.

\section{What Remains}

The next step is, as marked on the outline, to show that tight subtyping still
respects subject reduction and normalization. From there, we can use the same
steps as \citet{rapoport2017} to show progress and preservation.

\newpage
\appendix

\section{\DOTw{} Full rules}\label{appendix:dotw-rules}

\begin{figure}[ht]
  \begin{mathpar}
    \inferrule*{\phantom{}}{\isctx{\varnothing}} \and
    \inferrule*{\isctx{\Gamma}\and\Gamma \vdash \iskd{K}}
      {\isctx{\Gamma, X:K}} \and
    \inferrule*{\isctx{\Gamma}\and\Gamma \vdash A : \TyKd}
      {\isctx{\Gamma, x:A}} \and
  \end{mathpar}
  \caption{Context formation}
\end{figure}

\begin{figure}[ht]
  \begin{mathpar}
    \inferrule*[right=Wf-Intv]
      {\Gamma \vdash S : \TyKd \and \Gamma \vdash U : \TyKd}
      {\Gamma \vdash \iskd{\interval{S}{U}}} \and
    \inferrule*[right=Wf-DArr]
      {\Gamma \vdash \iskd{J} \and \Gamma, X:J \vdash \iskd{K}}
      {\Gamma \vdash \iskd{\KDepArr{X}{J}{K}}}
  \end{mathpar}
  \caption{Kind formation}
\end{figure}

\begin{figure}[ht]
  \begin{mathpar}
    \inferrule*[right=SK-Intv]
      {\Gamma \vdash S_2 \le S_1:\TyKd \and \Gamma \vdash U_1 \le U_2:\TyKd}
      {\Gamma \vdash \interval{S_1}{U_1} \le \interval{S_2}{U_2}} \and
    \inferrule*[right=SK-DArr]
      {\Gamma \vdash \iskd{\KDepArr{X}{J_1}{K_1}} \and
       \Gamma \vdash J_2 \le J_1 \and
       \Gamma, X:J_2 \vdash K_1 \le K_2}
      {\Gamma \vdash \KDepArr{X}{J_1}{K_1} \le \KDepArr{X}{J_2}{K_2}}
  \end{mathpar}
  \caption{Subkinding}
\end{figure}

\begin{figure}[ht]
  \begin{mathpar}
    \inferrule*[right=K-Var]{\isctx{\Gamma, X:K}}{\Gamma, X:K \vdash X:K} \and
    \inferrule*[right=K-Top]{\phantom{}}{\Gamma \vdash \top : \TyKd} \and
    \inferrule*[right=K-Bot]{\phantom{}}{\Gamma \vdash \bot : \TyKd} \and
    \inferrule*[right=K-Sing]{\Gamma \vdash A : \interval{S}{U}}
      {\Gamma \vdash A : \interval{A}{A}} \and
    \inferrule*[right=K-Arr]
      {\Gamma \vdash A : \TyKd \and \Gamma, x : A \vdash B : \TyKd}
      {\Gamma \vdash \TDepArr{x}{A}{B} : \TyKd} \and
    \inferrule*[right=K-Abs]
      {\Gamma \vdash \iskd{J} \and \Gamma, X:J \vdash A : K \and
       \Gamma, X:J \vdash \iskd{K}}
      {\Gamma \vdash \lambda(X:J).A : \KDepArr{X}{J}{K}} \and
    \inferrule*[right=K-App]
      {\Gamma \vdash A : \KDepArr{X}{J}{K} \and \Gamma \vdash B : J \and
       \Gamma, X:J \vdash \iskd{K} \and \Gamma \vdash \iskd{\subst{K}{B}{X}}}
      {\Gamma \vdash A\ B : \subst{K}{B}{X}} \and
    \inferrule*[right=K-Intersect]
      {\Gamma \vdash A : \interval{S_1}{U_1} \and
       \Gamma \vdash B : \interval{S_2}{U_2}}
      {\Gamma \vdash A \land B : \interval{S_1 \lor S_2}{U_1 \land U_2}} \and
    \inferrule*[right=K-Field]
      {\Gamma \vdash A : \interval{S}{U}}
      {\Gamma \vdash \objtyp{val}{\ell}{A} : *} \and
    \inferrule*[right=K-Typ]
      {\Gamma \vdash \iskd{K}}
      {\Gamma \vdash \objtyp{type}{M}{K} : *} \and
    \inferrule*[right=K-Typ-Mem]
      {\Gamma \vdash x : \objtyp{type}{M}{K}}
      {\Gamma \vdash x.M : K} \and
    \inferrule*[right=K-Rec]
      {\Gamma, x : \tau \vdash \tau : K}
      {\Gamma \vdash \mu(x . \tau) : K} \and
    \inferrule*[right=K-Sub]
      {\Gamma \vdash A:J \and \Gamma \vdash J \le K}{\Gamma \vdash A:K}
  \end{mathpar}
  \caption{Kind assignment}
\end{figure}
\clearpage

Note that \textsc{K-Intersect} rules refers to the union type $S_1 \lor S_2$,
despite no such construct being present in the language as a whole. I am
currently investigating whether the explicit addition of this construct is
necessary.

\begin{figure}[ht]
  \begin{mathpar}
    \inferrule*[right=ST-refl]{\Gamma \vdash A : K}
      {\Gamma \vdash A \le A : K} \and
    \inferrule*[right=ST-trans]
      {\Gamma \vdash A \le B : K \and
       \Gamma \vdash B \le C : K}
      {\Gamma \vdash A \le C : K} \and
    \inferrule*[right=ST-top]{\Gamma \vdash A : \interval{S}{U}}
      {\Gamma \vdash A \le \top : \TyKd} \and
    \inferrule*[right=ST-bot]{\Gamma \vdash A : \interval{S}{U}}
      {\Gamma \vdash \bot \le A : \TyKd} \and
    \inferrule*[right=ST-and-$\ell_1$]
      {\Gamma \vdash A \land B : K}{\Gamma \vdash A \land B \le A : K} \and
    \inferrule*[right=ST-and-$\ell_2$]
      {\Gamma \vdash A \land B : K}{\Gamma \vdash A \land B \le B : K} \and
    \inferrule*[right=ST-and-r]
      {\Gamma \vdash S \le A : K \and \Gamma \vdash S \le B : K}
      {\Gamma \vdash S \le A \land B : K} \and
    \inferrule*[right=ST-field]{\Gamma \vdash A \le B : \TyKd}
      {\Gamma \vdash \objtyp{val}{\ell}{A} \le \objtyp{val}{\ell}{B} : \TyKd}
      \and
    \inferrule*[right=ST-typ]
      {\Gamma \vdash J \le K}
      {\Gamma \vdash \objtyp{type}{M}{J} \le \objtyp{type}{M}{K} : \TyKd}\and
    \inferrule*[right=ST-$\beta_1$]
      {\Gamma Z : J \vdash A : K \and
       \Gamma \vdash B : J}
      {\Gamma \vdash (\lambda (Z: J).A)\ B \le \subst{A}{Z}{B} : \subst{K}{Z}{B}}
      \and
    \inferrule*[right=ST-$\beta_2$]
      {\Gamma Z : J \vdash A : K \and
       \Gamma \vdash B : J}
      {\Gamma \vdash \subst{A}{Z}{B} \le (\lambda (Z: J).A)\ B : \subst{K}{Z}{B}}
  \end{mathpar}
  \caption{Subtyping}
\end{figure}

\begin{figure}[ht]
  \begin{mathpar}
    \inferrule*[right=Eq]
      {\Gamma \vdash A \le B : K \and \Gamma \vdash B \le A : K}
      {\Gamma \vdash A = B : K}
  \end{mathpar}
  \caption{Type equality}
\end{figure}

\begin{figure}[ht]
  \begin{mathpar}
    \inferrule*[right=Ty-Var]
      {\isctx{\Gamma, x: \tau}}
      {\Gamma, x:\tau \vdash x: \tau} \and
    \inferrule*[right=Ty-Let]
      {\Gamma \vdash e_1: \tau \and \Gamma, x:\tau \vdash e_2 : \rho \and x \not\in fv(\rho)}
      {\Gamma \vdash \termlet{x}{e_1}{e_2} : \rho} \and
    \inferrule*[right=Ty-Fun-I]
      {\Gamma, x: \tau \vdash e: \rho}
      {\Gamma \vdash \lambda(x:\tau).e : \TDepArr{x}{\tau}{\rho}} \and
    \inferrule*[right=Ty-Fun-E]
      {\Gamma \vdash x : \TDepArr{z}{\tau}{\rho} \and \Gamma \vdash y : \tau}
      {\Gamma \vdash x\ y : \subst{\rho}{z}{y}} \and
    \inferrule*[right=Ty-$\mu$-I]
      {\Gamma \vdash x: \tau}
      {\Gamma \vdash x: \mu(x: \tau)} \and
    \inferrule*[right=Ty-$\mu$-E]
      {\Gamma \vdash x: \mu(z: \tau)}
      {\Gamma \vdash x: \subst{\tau}{z}{x}} \and
    \inferrule*[right=Ty-Rec-I]
      {\Gamma, x: \tau \vdash d: \tau}
      {\Gamma \vdash \nu(x: \tau)d : \mu(x:\tau)} \and
    \inferrule*[right=Ty-Rec-E]
      {\Gamma, x: \objtyp{val}{\ell}{\tau}}
      {\Gamma \vdash x.\ell : \tau} \and
    \inferrule*[right=Ty-And-I]
      {\Gamma \vdash x: \tau_1 \and \Gamma \vdash x: \tau_2}
      {\Gamma \vdash x: \tau_1 \land \tau_2} \and
    \inferrule*[right=Ty-Sub]
      {\Gamma \vdash e: \tau_1 \and \Gamma \vdash \tau_1 \le \tau_2 : *}
      {\Gamma \vdash e: \tau_2} \and
    \inferrule*[right=Ty-Def-Trm]
      {\Gamma \vdash e: \rho}
      {\Gamma \vdash \objval{val}{\ell}{e} : \objtyp{val}{\ell}{\rho}} \and
    \inferrule*[right=Ty-Def-Typ]
      {\Gamma \vdash \tau : K}
      {\Gamma \vdash \objval{type}{M}{A} : \objtyp{type}{M}{S(A:K)}}
  \end{mathpar}
  \caption{Type assignment}
\end{figure}

\section{Full tight typing rules}\label{appendix:tight-rules}

In most cases, tight typing is merely forwarded to the premises. In any rule
that extends the context with possibly-untrusted bounds, tight typing reverts
to general typing.

\begin{figure}[ht]
  \begin{mathpar}
    \inferrule*{\phantom{}}{\istctx{\varnothing}} \and
    \inferrule*{\istctx{\Gamma}\and\Gamma \vdash_\# \iskd{K}}
      {\isctx{\Gamma, X:K}} \and
    \inferrule*{\istctx{\Gamma}\and\Gamma \vdash_\# A : \TyKd}
      {\isctx{\Gamma, x:A}} \and
  \end{mathpar}
  \caption{Context formation}
\end{figure}

\begin{figure}[ht]
  \begin{mathpar}
    \inferrule*[right=Wf-Int-\#]
      {\Gamma \vdash_\# S : \TyKd \and \Gamma \vdash_\# U : \TyKd}
      {\Gamma \vdash_\# \iskd{\interval{S}{U}}} \and
    \inferrule*[right=Wf-DArr-\#]
      {\Gamma \vdash_\# \iskd{J} \and \Gamma, X:J \vdash_\# \iskd{K}}
      {\Gamma \vdash_\# \iskd{\KDepArr{X}{J}{K}}}
  \end{mathpar}
  \caption{Kind formation}
\end{figure}

\begin{figure}[ht]
  \begin{mathpar}
    \inferrule*[right=SK-Intv-\#]
      {\Gamma \vdash_\# S_2 \le S_1:\TyKd \and \Gamma \vdash_\# U_1 \le U_2:\TyKd}
      {\Gamma \vdash_\# \interval{S_1}{U_1} \le \interval{S_2}{U_2}} \and
    \inferrule*[right=SK-DArr-\#]
      {\Gamma \vdash_\# \iskd{\KDepArr{X}{J_1}{K_1}} \and
       \Gamma \vdash_\# J_2 \le J_1 \and
       \Gamma, X:J_2 \vdash K_1 \le K_2}
      {\Gamma \vdash_\# \KDepArr{X}{J_1}{K_1} \le \KDepArr{X}{J_2}{K_2}}
  \end{mathpar}
  \caption{Subkinding}
\end{figure}

\begin{figure}[ht]
  \begin{mathpar}
    \inferrule*[right=K-Var-\#]{\istctx{\Gamma, X:K}}{\Gamma, X:K \vdash_\# X:K} \and
    \inferrule*[right=K-Top-\#]{\phantom{}}{\Gamma \vdash_\# \top : \TyKd} \and
    \inferrule*[right=K-Bot-\#]{\phantom{}}{\Gamma \vdash_\# \bot : \TyKd} \and
    \inferrule*[right=K-Sing-\#]{\Gamma \vdash_\# A : \interval{S}{U}}
      {\Gamma \vdash_\# A : \interval{A}{A}} \and
    \inferrule*[right=K-Arr-\#]
      {\Gamma \vdash_\# A : \TyKd \and \Gamma, x : A \vdash B : \TyKd}
      {\Gamma \vdash_\# \TDepArr{x}{A}{B} : \TyKd} \and
    \inferrule*[right=K-Abs-\#]
      {\Gamma \vdash_\# \iskd{J} \and \Gamma, X:J \vdash A : K \and
       \Gamma, X:J \vdash_\# \iskd{K}}
      {\Gamma \vdash_\# \lambda(X:J).A : \KDepArr{X}{J}{K}} \and
    \inferrule*[right=K-App-\#]
      {\Gamma \vdash_\# A : \KDepArr{X}{J}{K} \and \Gamma \vdash_\# B : J \and
       \Gamma, X:J \vdash \iskd{K} \and \Gamma \vdash_\# \iskd{\subst{K}{B}{X}}}
      {\Gamma \vdash_\# A\ B : \subst{K}{B}{X}} \and
    \inferrule*[right=K-Intersect-\#]
      {\Gamma \vdash_\# A : \interval{S_1}{U_1} \and
       \Gamma \vdash_\# B : \interval{S_2}{U_2}}
      {\Gamma \vdash_\# A \land B : \interval{S_1 \lor S_2}{U_1 \land U_2}} \and
    \inferrule*[right=K-Field-\#]
      {\Gamma \vdash_\# A : \interval{S}{U}}
      {\Gamma \vdash_\# \objtyp{val}{\ell}{A} : *} \and
    \inferrule*[right=K-Typ-\#]
      {\Gamma \vdash_\# \iskd{K}}
      {\Gamma \vdash_\# \objtyp{type}{M}{K} : *} \and
    \inferrule*[right=K-Typ-Mem-\#]
      {\Gamma \vdash_! x : \objtyp{type}{M}{S(A:K)}}
      {\Gamma \vdash_\# x.M : S(A:K)} \and
    \inferrule*[right=K-Rec-\#]
      {\Gamma, x : \tau \vdash \tau : K}
      {\Gamma \vdash_\# \mu(x . \tau) : K} \and
    \inferrule*[right=K-Sub-\#]
      {\Gamma \vdash_\# A:J \and \Gamma \vdash_\# J \le K}{\Gamma \vdash_\# A:K}
  \end{mathpar}
  \caption{Kind assignment}
\end{figure}

\begin{figure}[ht]
  \begin{mathpar}
    \inferrule*[right=ST-refl-\#]{\Gamma \vdash_\# A : K}
      {\Gamma \vdash_\# A \le A : K} \and
    \inferrule*[right=ST-trans-\#]
      {\Gamma \vdash_\# A \le B : K \and
       \Gamma \vdash_\# B \le C : K}
      {\Gamma \vdash_\# A \le C : K} \and
    \inferrule*[right=ST-top-\#]{\Gamma \vdash_\# A : \interval{S}{U}}
      {\Gamma \vdash_\# A \le \top : \TyKd} \and
    \inferrule*[right=ST-bot-\#]{\Gamma \vdash_\# A : \interval{S}{U}}
      {\Gamma \vdash_\# \bot \le A : \TyKd} \and
    \inferrule*[right=ST-and-$\ell_1$-\#]
      {\Gamma \vdash_\# A \land B : K}{\Gamma \vdash_\# A \land B \le A : K} \and
    \inferrule*[right=ST-and-$\ell_2$-\#]
      {\Gamma \vdash_\# A \land B : K}{\Gamma \vdash_\# A \land B \le B : K} \and
    \inferrule*[right=ST-and-r-\#]
      {\Gamma \vdash_\# S \le A : K \and \Gamma \vdash_\# S \le B : K}
      {\Gamma \vdash_\# S \le A \land B : K} \and
    \inferrule*[right=ST-field-\#]{\Gamma \vdash_\# A \le B : \TyKd}
      {\Gamma \vdash_\# \objtyp{val}{\ell}{A} \le \objtyp{val}{\ell}{B} : \TyKd}
      \and
    \inferrule*[right=ST-typ-\#]
      {\Gamma \vdash_\# J \le K}
      {\Gamma \vdash_\# \objtyp{type}{M}{J} \le \objtyp{type}{M}{K} : \TyKd}\and
    \inferrule*[right=ST-$\beta_1$-\#]
      {\Gamma \vdash_\# B : J \and
       \Gamma, Z: S(B:J) \vdash_\# A : K}
      {\Gamma \vdash_\# (\lambda (Z: J).A)\ B \le \subst{A}{Z}{B} : \subst{K}{Z}{B}}
      \and
    \inferrule*[right=ST-$\beta_2$-\#]
      {\Gamma \vdash_\# B : J \and
       \Gamma, Z: S(B:J) \vdash_\# A : K}
      {\Gamma \vdash_\# \subst{A}{Z}{B} \le (\lambda (Z: J).A)\ B : \subst{K}{Z}{B}}
  \end{mathpar}
  \caption{Subtyping}
\end{figure}

\begin{figure}[ht]
  \begin{mathpar}
    \inferrule*[right=Eq-\#]
      {\Gamma \vdash_\# A \le B : K \and \Gamma \vdash_\# B \le A : K}
      {\Gamma \vdash_\# A = B : K}
  \end{mathpar}
  \caption{Type equality}
\end{figure}

\begin{figure}[ht]
  \begin{mathpar}
    \inferrule*[right=Ty-Var-\#]
      {\istctx{\Gamma, x: \tau}}
      {\Gamma, x:\tau \vdash_\# x: \tau} \and
    \inferrule*[right=Ty-Let-\#]
      {\Gamma \vdash_\# e_1: \tau \and \Gamma, x:\tau \vdash e_2: \rho \and x \not\in fv(\rho)}
      {\Gamma \vdash_\# \termlet{x}{e_1}{e_2} : \rho} \and
    \inferrule*[right=Ty-Fun-I-\#]
      {\Gamma, x: \tau \vdash e: \rho}
      {\Gamma \vdash_\# \lambda(x:\tau).e : \TDepArr{x}{\tau}{\rho}} \and
    \inferrule*[right=Ty-Fun-E-\#]
      {\Gamma \vdash_\# x : \TDepArr{z}{\tau}{\rho} \and \Gamma \vdash_\# y : \tau}
      {\Gamma \vdash_\# x\ y : \subst{\rho}{z}{y}} \and
    \inferrule*[right=Ty-$\mu$-I-\#]
      {\Gamma \vdash_\# x: \tau}
      {\Gamma \vdash_\# x: \mu(x: \tau)} \and
    \inferrule*[right=Ty-$\mu$-E-\#]
      {\Gamma \vdash_\# x: \mu(z: \tau)}
      {\Gamma \vdash_\# x: \subst{\tau}{z}{x}} \and
    \inferrule*[right=Ty-Rec-I-\#]
      {\Gamma, x: \tau \vdash d: \tau}
      {\Gamma \vdash_\# \nu(x: \tau)d : \mu(x:\tau)} \and
    \inferrule*[right=Ty-Rec-E-\#]
      {\Gamma, x: \objtyp{val}{\ell}{\tau}}
      {\Gamma \vdash_\# x.\ell : \tau} \and
    \inferrule*[right=Ty-And-I-\#]
      {\Gamma \vdash_\# x: \tau_1 \and \Gamma \vdash_\# x: \tau_2}
      {\Gamma \vdash_\# x: \tau_1 \land \tau_2} \and
    \inferrule*[right=Ty-Sub-\#]
      {\Gamma \vdash_\# e: \tau_1 \and \Gamma \vdash_\# \tau_1 \le \tau_2 : *}
      {\Gamma \vdash_\# e: \tau_2} \and
    \inferrule*[right=Ty-Def-Trm-\#]
      {\Gamma \vdash_\# e: \rho}
      {\Gamma \vdash_\# \objval{val}{\ell}{e} : \objtyp{val}{\ell}{\rho}} \and
    \inferrule*[right=Ty-Def-Typ-\#]
      {\Gamma \vdash_\# \tau : K}
      {\Gamma \vdash_\# \objval{type}{M}{A} : \objtyp{type}{M}{S(A:K)}}
  \end{mathpar}
  \caption{Type assignment}
\end{figure}

\begin{figure}
  \begin{mathpar}
    \inferrule*[right=Var-!]{\phantom{}}{\Gamma, x:\tau \vdash_! x:\tau}\and
    \inferrule*[right=Rec-E-!]{\Gamma \vdash_! x: \mu(z:\tau)}
      {\Gamma \vdash_! x: \subst{\tau}{z}{x}} \and
    \inferrule*[right=And$_1$-E-!]
      {\Gamma \vdash_! x: \tau_1 \land \tau_2}
      {\Gamma \vdash_! x: \tau_1}\and
    \inferrule*[right=And$_2$-E-!]
      {\Gamma \vdash_! x: \tau_1 \land \tau_2}
      {\Gamma \vdash_! x: \tau_2}\and
    \inferrule*[right=Fun-I-!]
      {\Gamma, x: \tau \vdash e: \rho \and x \not\in fv(\tau)}
      {\Gamma \vdash_! \lambda(x:\tau).e : \TDepArr{x}{\tau}{\rho}} \and
    \inferrule*[right=Record-I-!]
      {\Gamma, x: \tau \vdash d: \tau}
      {\Gamma \vdash_! \nu(x:\tau)d : \mu(x:\tau)}
  \end{mathpar}
  \caption{Precise Value and Variable Typing}
\end{figure}

\end{document}

