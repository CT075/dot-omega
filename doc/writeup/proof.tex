\documentclass[a4paper, 10pt]{article}
\usepackage{amssymb, amsmath, amsthm}
\usepackage[backend=bibtex,natbib=true]{biblatex}
\usepackage{enumitem}
\usepackage{mathrsfs}
\usepackage{parskip}
\usepackage{mathpartir}
\usepackage{bussproofs}
\usepackage{stmaryrd}
\usepackage{appendix}
\usepackage{datetime}
\usepackage{tabularx}
\usepackage[iso, english]{isodate}

\newcommand{\DOTw}{\ensuremath{DOT^\omega}}
\newcommand{\Dsub}{\ensuremath{D_{<:}}}
\newcommand{\interval}[2]{#1 .. #2}
\newcommand{\isctx}[1]{#1\ \texttt{ctx}}
\newcommand{\iskd}[1]{#1\ \texttt{kd}}
\newcommand{\TyKd}{*}
\newcommand{\KDepArr}[3]{\Pi(#1:#2).#3}
\newcommand{\TDepArr}[3]{(#1:#2) \rightarrow #3}
\newcommand{\subst}[3]{#1[#2/#3]}
\newcommand{\objtyp}[3]{\{ \textbf{#1}\ #2 : #3 \}}
\newcommand{\objval}[3]{\{ \textbf{#1}\ #2 = #3 \}}
\newcommand{\termlet}[3]{\text{let }#1 = #2\text{ in }#3}

\newtheorem{theorem}{Theorem}
\newtheorem{defn}{Definition}
\newtheorem{lemma}{Lemma}
\newtheorem{property}{Property}

\bibliography{paper}

\setlength{\parindent}{0cm}

\title{Soundness for \DOTw{} (update \today)}
\author{Cameron Wong}

\begin{document}
\maketitle

\setlength{\parskip}{\baselineskip}

\section{Background}

The key difficulty in introducing higher kinded types to DOT is, like with any
DOT soundness proof, bad bounds. Particularly, in an open context, neither
types nor terms satisfy subject reduction, making it unsound to perform
evaluation in the presence of arbitrary subtyping lattices. This phenomenon
is well-documented by \citet{amin2016}, \citet{rapoport2017},
\citet{stucki2021} and others besides.

The usual approach to dealing with this problem is to consider soundness only
in closed terms, or in limited contexts where all bounds are known to be good.
This notion was originally formalized by \citet{amin2016} as \emph{tight
typing}. \citet{rapoport2017} then demonstrated how tight typing can be used to
dramatically simplify the proof of soundness for DOT. This was done by defining
\emph{inert typing contexts}, in which tight typing and general typing for DOT
are equivalent. This allowed them to prove properties of the overall DOT type
system by first dropping into tight typing, then using simple, intuitive
reasoning over the tightly-typed system. In the words of \citet{rapoport2017},
``To reconcile a subtyping lattice with a sound language, we only need to force
the programmer to provide evidence that the custom lattice does not violate any
familiar assumptions... This evidence takes the form of an argument value that
must be passed to the lambda before the body that uses the custom type lattice
can be allowed to execute.''.

A major difficulty in extending this recipe to cover higher-kinded types is
that, when types can contain computation themselves, some $\beta$-reduction
\emph{has} to be performed in the typing rules (e.g., to compare types during
typechecking), which is similarly unsound in the presence of bad bounds.
Thankfully, this, too, can be resolved by making the following observation:

\begin{property}
  If $\varnothing \vdash e : \tau$, then $\varnothing \vdash \tau : *$.
\end{property}

That is, any closed expression $e$ must be assigned a proper type (as opposed
to a type lambda) if it has any type at all. This means that, even if $\tau$ is
a complex type expression containing many type functions, each of those
functions must ultimately be applied to another type expression witnessing the
validity of any bounds introduced. We will formally define this property in
Section \ref{sec:tight-typing}.

\section{$\DOTw$ Syntax}

\begin{figure}[ht]
  \centering
  \begin{tabularx}{\linewidth}{XXXX}
    $x,y,z$... & \textbf{Term Variable} & $X,Y,Z$... & \textbf{Type Variable}\\
    $\ell$ & \textbf{Value Label} & $M$ & \textbf{Type Label} \\
  \end{tabularx}
  \hfill \\
  \hfill \\
  \begin{tabularx}{\linewidth}{XlX}
    $e,t ::=$ &
    $x \mid v \mid x.\ell \mid x\ y \mid \termlet{x}{e}{t}$
      & \textbf{Term} \\
    $v ::=$ & $\lambda(x:\tau).e \mid \nu(x:\tau).d$ & \textbf{Value} \\
    $d ::=$ & $\objval{val}{\ell}{e} \mid \objval{type}{M}{A} \mid
      d_1 \land d_2$ & \textbf{Definition} \\
    $A,B,C ::=$ &
    $X \mid \lambda(X:K).A \mid x.M \mid A\ B$ & \textbf{Type} \\
    $(\tau, \rho, S, U ::=)$ &
      $\mid \top \mid \bot \mid \TDepArr{x}{\tau}{\rho}
      \mid \tau \land \rho \mid \mu(x.\tau)$ & (Proper types) \\
    & $\mid \objtyp{val}{\ell}{\tau} \mid \objtyp{type}{M}{K}$ & \\
    $J,K ::=$ & $\interval{S}{U} \mid \KDepArr{X}{J}{K}$ & \textbf{Kind} \\
  \end{tabularx}
  \caption{Abstract syntax of $\DOTw{}$}
\end{figure}

We also use $*$ as shorthand for $\interval{\bot}{\top}$.

Following \citet{stucki2021}, we represent type bounds at the kind level, with
$\Gamma \vdash \tau : \interval{S}{U}$ meaning $S \le \tau$ and $\tau \le U$.
Type members are then assigned a kind, rather than marking bounds
syntactically.

The full declarative typing system can be found in Appendix
\ref{appendix:dotw-rules}.

\section{The Proof}

\subsection{Overview}

Our general proof recipe will largely follow that of \citet{rapoport2017}.
Unlike the first-order system, merely knowing that $\Gamma \vdash v: A$ does
not allow us to deduce the canonical form of the value $v$, as $A$ may be a
complex type expression containing $\beta$-redexes. Nor can we simply normalize
the type expression $A$, as we may be in a crazy context in which subtyping
cannot be trusted and subject reduction does not hold. If we \emph{do} know
that the context is inert, however, then reduction is sound and all is well.

The overall proof, then, will proceed as follows:

\begin{enumerate}
  \item Extend DOT's tight typing rules with corresponding tight \emph{kinding}
    and reduction rules, alongside extending the notion of inert contexts to
    restrict type variables as well.
  \item Show that general typing still implies tight typing.
  \item Prove that, under tight subtyping, reduction of type expressions is
    sound and normalizing.
  \item Finally, use the above to re-prove the relevant canonical forms lemmas
    necessary to show progress and preservation of $\DOTw$.
\end{enumerate}

\subsection{Tight Kinding and Reduction}\label{sec:tight-typing}

\appendix

\section{\DOTw{} Full rules}\label{appendix:dotw-rules}

\begin{figure}[ht]
  \begin{mathpar}
    \inferrule*{\phantom{}}{\isctx{\varnothing}} \and
    \inferrule*{\isctx{\Gamma}\and\Gamma \vdash \iskd{K}}
      {\isctx{\Gamma, X:K}} \and
    \inferrule*{\isctx{\Gamma}\and\Gamma \vdash A : \TyKd}
      {\isctx{\Gamma, x:A}} \and
  \end{mathpar}
  \caption{Context formation}
\end{figure}

\begin{figure}[ht]
  \begin{mathpar}
    \inferrule*[right=Wf-Intv]
      {\Gamma \vdash S : \TyKd \and \Gamma \vdash U : \TyKd}
      {\Gamma \vdash \iskd{\interval{S}{U}}} \and
    \inferrule*[right=Wf-DArr]
      {\Gamma \vdash \iskd{J} \and \Gamma, X:J \vdash \iskd{K}}
      {\Gamma \vdash \iskd{\KDepArr{X}{J}{K}}}
  \end{mathpar}
  \caption{Kind formation}
\end{figure}

\begin{figure}[ht]
  \begin{mathpar}
    \inferrule*[right=SK-Intv]
      {\Gamma \vdash S_2 \le S_1:\TyKd \and \Gamma \vdash U_1 \le U_2:\TyKd}
      {\Gamma \vdash \interval{S_1}{U_1} \le \interval{S_2}{U_2}} \and
    \inferrule*[right=SK-DArr]
      {\Gamma \vdash \iskd{\KDepArr{X}{J_1}{K_1}} \and
       \Gamma \vdash J_2 \le J_1 \and
       \Gamma, X:J_2 \vdash K_1 \le K_2}
      {\Gamma \vdash \KDepArr{X}{J_1}{K_1} \le \KDepArr{X}{J_2}{K_2}}
  \end{mathpar}
  \caption{Subkinding}
\end{figure}

\begin{figure}[ht]
  \begin{mathpar}
    \inferrule*[right=K-Var]{\isctx{\Gamma, X:K}}{\Gamma, X:K \vdash X:K} \and
    \inferrule*[right=K-Top]{\phantom{}}{\Gamma \vdash \top : \TyKd} \and
    \inferrule*[right=K-Bot]{\phantom{}}{\Gamma \vdash \bot : \TyKd} \and
    \inferrule*[right=K-Sing]{\Gamma \vdash A : \interval{S}{U}}
      {\Gamma \vdash A : \interval{A}{A}} \and
    \inferrule*[right=K-Arr]
      {\Gamma \vdash A : \TyKd \and \Gamma, x : A \vdash B : \TyKd}
      {\Gamma \vdash \TDepArr{x}{A}{B} : \TyKd} \and
    \inferrule*[right=K-Abs]
      {\Gamma \vdash \iskd{J} \and \Gamma, X:J \vdash A : K \and
       \Gamma, X:J \vdash \iskd{K}}
      {\Gamma \vdash \lambda(X:J).A : \KDepArr{X}{J}{K}} \and
    \inferrule*[right=K-App]
      {\Gamma \vdash A : \KDepArr{Z}{J}{K} \and \Gamma \vdash B : J \and
       \Gamma, X:J \vdash \iskd{K} \and \Gamma \vdash \iskd{\subst{K}{B}{X}}}
      {\Gamma \vdash A B : \subst{K}{B}{X}} \and
    \inferrule*[right=K-Intersect]
      {\Gamma \vdash A : \interval{S_1}{U_1} \and
       \Gamma \vdash B : \interval{S_2}{U_2}}
      {\Gamma \vdash A \land B : \interval{S_1 \lor S_2}{U_1 \land U_2}} \and
    \inferrule*[right=K-Field]
      {\Gamma \vdash A : \interval{S}{U}}
      {\Gamma \vdash \objtyp{val}{\ell}{A}} \and
    \inferrule*[right=K-Typ]
      {\Gamma \vdash \iskd{K}}
      {\Gamma \vdash \objtyp{type}{M}{K}} \and
    \inferrule*[right=K-Typ-Mem]
      {\Gamma \vdash x : \objtyp{type}{M}{K}}
      {\Gamma \vdash x.M : K} \and
    \inferrule*[right=K-Rec]
      {\Gamma, x : \tau \vdash \tau : K}
      {\Gamma \vdash \mu(x . \tau) : K} \and
    \inferrule*[right=K-Sub]
      {\Gamma \vdash A:J \and \Gamma \vdash J \le K}{\Gamma \vdash A:K}
  \end{mathpar}
  \caption{Kind assignment}
\end{figure}

\begin{figure}[ht]
  \begin{mathpar}
    \inferrule*[right=ST-refl]{\Gamma \vdash A : K}
      {\Gamma \vdash A \le A : K} \and
    \inferrule*[right=ST-trans]
      {\Gamma \vdash A \le B : K \and
       \Gamma \vdash B \le C : K}
      {\Gamma \vdash A \le C : K} \and
    \inferrule*[right=ST-top]{\Gamma \vdash A : \interval{S}{U}}
      {\Gamma \vdash A \le \top : \TyKd} \and
    \inferrule*[right=ST-bot]{\Gamma \vdash A : \interval{S}{U}}
      {\Gamma \vdash \bot \le A : \TyKd} \and
    \inferrule*[right=ST-and-$\ell_1$]
      {\Gamma \vdash A \land B : K}{\Gamma \vdash A \land B \le A : K} \and
    \inferrule*[right=ST-and-$\ell_2$]
      {\Gamma \vdash A \land B : K}{\Gamma \vdash A \land B \le B : K} \and
    \inferrule*[right=ST-and-r]
      {\Gamma \vdash S \le A : K \and \Gamma \vdash S \le B : K}
      {\Gamma \vdash S \le A \land B : K} \and
    \inferrule*[right=ST-field]{\Gamma \vdash A \le B : K}
      {\Gamma \vdash \objtyp{val}{\ell}{A} \le \objtyp{val}{\ell}{B} : \TyKd}
      \and
    \inferrule*[right=ST-typ]
      {\Gamma \vdash J \le K}
      {\Gamma \vdash \objtyp{type}{M}{J} \le \objtyp{type}{M}{K} : \TyKd}\and
    \inferrule*[right=ST-$\beta_1$]
      {\Gamma \vdash X = \lambda (Z: J).A : \KDepArr{Z}{J}{K} \and
       \Gamma \vdash Y : J}
      {\Gamma \vdash X\ Y \le \subst{A}{Y}{Z} : \subst{K}{Y}{Z}}\and
    \inferrule*[right=ST-$\beta_2$]
      {\Gamma \vdash X = \lambda (Z: J).A : \KDepArr{Z}{J}{K} \and
       \Gamma \vdash Y : J}
      {\Gamma \vdash \subst{A}{Y}{Z} \le X\ Y : \subst{K}{Y}{Z}}\and
    \inferrule*[right=ST-$\eta_1$]
      {\Gamma \vdash A : \KDepArr{Z}{J}{K} \and Z \not\in fv(A)}
      {\Gamma \vdash \lambda (Z:J).A\ Z \le A: \KDepArr{Z}{J}{K}} \and
    \inferrule*[right=ST-$\eta_2$]
      {\Gamma \vdash A : \KDepArr{Z}{J}{K} \and Z \not\in fv(A)}
      {\Gamma \vdash \lambda (Z:J).A \le A\ Z: \KDepArr{Z}{J}{K}}
  \end{mathpar}
  \caption{Subtyping}
\end{figure}

\begin{figure}[ht]
  \begin{mathpar}
    \inferrule*[right=Eq]
      {\Gamma \vdash A \le B : K \and \Gamma \vdash B \le A : K}
      {\Gamma \vdash A = B : K}
  \end{mathpar}
  \caption{Type equality}
\end{figure}

\begin{figure}[ht]
  \begin{mathpar}
    \inferrule*[right=Ty-Var]
      {\isctx{\Gamma, x: \tau}}
      {\Gamma, x:\tau \vdash x: \tau} \and
    \inferrule*[right=Ty-Let]
      {\Gamma \vdash e_1: \tau \and \Gamma, x:\tau \in e_2 \and x \not\in fv(\rho)}
      {\Gamma \vdash \termlet{x}{e_1}{e_2} : \rho} \and
    \inferrule*[right=Ty-Fun-I]
      {\Gamma, x: \tau \vdash e: \rho}
      {\Gamma \vdash \lambda(x:\tau).e : \TDepArr{x}{\tau}{\rho}} \and
    \inferrule*[right=Ty-Fun-E]
      {\Gamma \vdash x : \TDepArr{z}{\tau}{\rho} \and \Gamma \vdash y : \tau}
      {\Gamma \vdash x\ y : \subst{z}{y}{\rho}} \and
    \inferrule*[right=Ty-$\mu$-I]
      {\Gamma \vdash x: \tau}
      {\Gamma \vdash x: \mu(x: \tau)} \and
    \inferrule*[right=Ty-$\mu$-E]
      {\Gamma \vdash x: \mu(z: \tau)}
      {\Gamma \vdash x: \subst{z}{x}{\tau}} \and
    \inferrule*[right=Ty-Rec-I]
      {\Gamma, x: \tau \vdash d: \tau}
      {\Gamma \vdash \nu(x: \tau)d : \mu(x:\tau)} \and
    \inferrule*[right=Ty-Rec-E]
      {\Gamma, x: \objtyp{val}{\ell}{\tau}}
      {\Gamma \vdash x.\ell : \tau} \and
    \inferrule*[right=Ty-And-I]
      {\Gamma \vdash x: \tau_1 \and \Gamma \vdash x: \tau_2}
      {\Gamma \vdash x: \tau_1 \land \tau_2} \and
    \inferrule*[right=Ty-Sub]
      {\Gamma \vdash e: \tau_1 \and \Gamma \vdash \tau_1 \le \tau_2 : *}
      {\Gamma \vdash e: \tau_2} \and
    \inferrule*[right=Ty-Def-Trm]
      {\Gamma \vdash e: \rho}
      {\Gamma \vdash \objval{val}{\ell}{e} : \objtyp{val}{\ell}{\rho}} \and
    \inferrule*[right=Ty-Def-Typ]
      {\Gamma \vdash \tau : K}
      {\Gamma \vdash \objval{type}{M}{\tau} : \objtyp{type}{M}{\interval{\tau}{\tau}}}
  \end{mathpar}
  \caption{Type assignment}
\end{figure}

\end{document}

